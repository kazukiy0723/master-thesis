\chapter{先行研究}
本章では,過去に行われた遅延聴覚フィードバックの影響を調査する研究について紹介する.
\section{遅延聴覚フィードバックが発話に与える影響の調査}
先行研究\cite{kayama}では,主観評価による遅延聴覚フィードバックが発話に与える影響の調査が行われた.
この主観評価は,耳介付近に伝達された音に一定の遅延を発生させて外耳道に出力する装置(以後,音響測定装置と呼ぶ)を装着した被験者が,
発話時の違和感を主観的に評価するという内容のものである.人は,自身の発話音声を聴取するタイミングに遅延が生じると,違和感を覚えて発話が阻害されると
考えられるため,このときの違和感を遅延時間ごとに主観的に評価することにより,音声における入出力信号間の遅延時間の許容範囲を検討することが可能になる.
この調査では,被験者が原稿を読み上げているとき,発話音声が音響測定装置を通して耳に戻るまでの時間を変化させ,
被験者が感じる違和感の程度を調査した.
被験者は,「読み上げるときにしゃべりにくくないか」および「遅れが気にならないか」の2つについて「優(4点)」「良(3点)」「可(2点)」「不可(1点)」の4段階で評価する.
評価基準は,表\ref{table:evaluation-1}および表\ref{table:evaluation-2}のとおりである.
得られた評価結果は被験者間で平均され,評点は,しゃべりやすさと遅延の気にならなさに比例する.
若年者と高齢者の間で評点を比較することで,発話音声に発生させる遅延時間の大きさと違和感の程度が,若年者と高齢者の間でどのように異なるかを観察した.
%%%% 評価基準の表
\begin{table}[tbp]
  \caption{読み上げるときにしゃべりにくくないかの評価基準}
  \label{table:evaluation-1}
  \centering
  \begin{tabular}{lllc}
    \hline
    評価 & 評価基準 & 評点\\
    \hline \hline
    優  & しゃべりにくくない & 4\\
    良  & しゃべりにくいが気にならない & 3\\
    可  & しゃべりにくい & 2\\
    不可  & とてもしゃべりにくい & 1\\
    \hline
  \end{tabular}
\end{table}
%%%%%%%%%%%%%%%%%%%%%%%%%%%%
\begin{table}[tbp]
  \caption{遅れが気にならないかの評価基準}
  \label{table:evaluation-2}
  \centering
  \begin{tabular}{lllc}
    \hline
    評価 & 評価基準 & 評点\\
    \hline \hline
    優  & 遅れがまったくわからない & 4\\
    良  & 遅れが分かるが気にならない & 3\\
    可  & 遅れが気になる & 2\\
    不可  & 遅れがはっきり分かる & 1\\
    \hline
  \end{tabular}
\end{table}
%%%%%%%%%%%%%%%%%%%%%%%%%%%%
また,文献\cite{shigematu-toukyoushibu}での調査結果によると,若年者と高齢者の間で聴覚フィードバックの遅延時間の許容量には差異があったものの,
両者で提示した遅延時間に違いがあったため,その差が統計的に有意であるかの分析が困難であった.
そのため,文献\cite{kayama}では両者の遅延時間を揃えるために高齢者に提示した遅延時間と同一の条件で若年者に対して主観評価実験を行った.
遅延時間ごとに若年者と高齢者の結果の分布をコルモゴロフ・スミノルフ検定で比較したところ,90[ms]以上の遅延で若年者と高齢者の遅延の
感じやすさに有意な差が存在することが示され,若年者は高齢者と比較して遅延の影響をより敏感に受けやすいという結果が得られた.
一方で,この調査では主観評価による個人差が比較的大きくなることも明らかとなった.
\section{遅延聴覚フィードバックが身体運動に与える影響の調査}
文献\cite{kayama}では,遅延聴覚フィードバックが発話に与える影響を検討したが,その評価は主観的な手法によるものであるため,個人差の影響を軽減する客観的な評価方法の必要性が指摘されている.
そこで,重松らによる研究\cite{shigematu}では,遅延聴覚フィードバックの影響を客観的に評価するために,
テンポの画面提示アプリケーション\cite{Syuuronn-shigematu}を活用したボタン押し課題を通じて,遅延聴覚フィードバックが身体運動に及ぼす影響の調査が行われた.
ボタン押し課題は,遅延聴覚フィードバックの下で,一定のテンポでボタンを押すことを被験者に要求する課題である.
テンポの画面表示アプリケーションは,画面の上部に表示される短いバーが画面下部の長いバーへ向けて一定速度で移動し,
両バーが重なるタイミングで被験者がボタンを押下するタイミングが示される仕組みである.
また,長いバーの点滅もタイミングの指示に利用された.
この課題では,指定された遅延時間を用いて若年者8名を対象に実験が実施された.
結果として,この課題による遅延聴覚フィードバックが身体運動に与える影響の観察が可能であることが分かった.その一方で,
被験者間で遅延時間への反応に差異が認められ,特定の被験者では遅延時間の増加がボタン押下時間間隔に及ぼす影響が限定的であることが確認された.
これは,被験者が遅延時間に関わらず聴覚フィードバックとしてボタン押下時の音を認識せずに画面のみに従ってボタン押し課題を行ってしまった可能性を示唆している.
したがって,被験者が一貫してヘッドホンからの音を聴覚フィードバックとして認識できるようにテンポの提示方法の改良が必要であることが明らかになった.
本研究は,このボタン押し課題のシステムを改良し,遅延聴覚フィードバックが身体運動に与える影響をより明確に観察することを目指している.
