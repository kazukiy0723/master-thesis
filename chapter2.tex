\chapter{先行研究}
本章では,過去に行われた遅延聴覚フィードバックの影響を調査する研究について紹介する.
\section{遅延聴覚フィードバックが発話に及ぼす影響の調査}
文献\cite{kayama}では,主観評価による遅延聴覚フィードバックの影響の調査が行われた.
主観調査とは,耳介付近に伝達された音に一定の遅延を発生させて外耳道に出力する装置(以下,音響測定装置と呼ぶ)を装着した被験者が,
発話したときの違和感を主観評価するという内容の調査である.
先行研究\cite{kayama}では,聞こえの調査によって若年者と高齢者の間で遅延時間の許容量に差があるという結果が得られたが,両者で提示した遅延時間に違いがあり,その差が統計的に有意と言えるかについて分析することができなかった.そこで,文献[香山さんの論文]では,両者の遅延時間を揃えるために高齢者に提示した遅延時間と同じもので若年者に対して主観評価実験を実施した.聞こえの調査で得られた高齢者と若年者の結果についてコルモゴロフ・スミノルフ検定を用いた検定では,若年者と高齢者での遅延の感じやすさに有意な差が存在することが示唆されており,若年者は高齢者と比較してその影響を敏感に受けやすいという結果が得られた.
\section{遅延聴覚フィードバックが身体運動に及ぼす影響の調査}
重松氏らによる研究\cite{shigematu}では,聴覚フィードバックの影響を客観的に評価するために,テンポの画面提示アプリケーションによるボタン押し課題(4章を参照)を用いた遅延聴覚フィードバックの身体運動への影響の調査が行われた.
この課題では,指定された遅延時間(2.1節参照)を用いて若年者8名に対する実験が行われた.
結果として,遅延時間に応じた被験者ごとの反応には差異が見られ,一部の被験者では遅延時間の増加に伴うボタン押下間隔の変動が少ないことが確認された.
これは,被験者が遅延時間に関わらず聴覚フィードバックとしてボタンの音を認識していない可能性があることが示唆されている.
したがって,全被験者がヘッドホンからの音を聴覚フィードバックとして認識するようにテンポ提示方法を改良することが求められる.
本研究では,このボタン押下課題のシステムを改良し,遅延聴覚フィードバックが身体運動に与える影響をより明確に観察できるようにすることを目的とする.
