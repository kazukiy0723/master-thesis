\chapter{先行研究}
本章では,過去に行われた遅延聴覚フィードバックの影響を調査する研究について紹介する.
\section{遅延聴覚フィードバックが発話に及ぼす影響の調査}
文献\cite{kayama}では,主観評価による遅延聴覚フィードバックの影響の調査が行われた.
主観調査とは,耳介付近に伝達された音に一定の遅延を発生させて外耳道に出力する装置(以下,音響測定装置と呼ぶ)を装着した被験者が,
発話したときの違和感を主観評価するという内容の調査である.
先行研究\cite{kayama}での調査結果によれば,若年者と高齢者の間で聴覚フィードバックの遅延時間の許容量に差異があったものの,
両者で提示した遅延時間に違いがあったため,その差が統計的に有意であるかの分析が困難であった.
そのため,著者らは,両者の遅延時間を揃えるために高齢者に提示した遅延時間と同一の条件で若年者に対して主観評価実験を実施した.
聞こえの調査で得られた高齢者と若年者の結果についてコルモゴロフ・スミノルフ検定を用いた検定では,90ms以上の遅延時間帯で若年者と高齢者での遅延の感じやすさに有意な差が存在することが示され,
若年者は高齢者と比較してその影響をより敏感に受けやすいという結果が得られた.
\section{遅延聴覚フィードバックが身体運動に及ぼす影響の調査}
重松氏らによる研究\cite{shigematu}では,聴覚フィードバックの影響を客観的に評価するために,
テンポの画面提示アプリケーション\cite{Syuuronn-shigematu}によるボタン押し課題(4章を参照)を用いた遅延聴覚フィードバックの身体運動への影響の調査が行われた.
テンポの画面表示アプリケーションは,画面の上部に表示される短いバーが画面の下部で静止している長いバーへ向けて一定速度で移動し,
被験者がボタンを押下すべきタイミングで2つのバー同士がぴったり重なる.また,そのタイミングで
長いバーは点滅する.この2つの動作により,被験者にボタンを押下すべきタイミングを提示する.
この課題では,指定された遅延時間を用いて若年者8名に対する実験が行われた.
結果として,遅延時間に応じた被験者ごとの反応には差異が見られ,一部の被験者では遅延時間の増加に伴うボタン押下間隔の変動が少ないことが確認された.
これは,被験者が遅延時間に関わらず聴覚フィードバックとしてボタンの音を認識していない可能性があることが示唆されている.
したがって,全被験者がヘッドホンからの音を聴覚フィードバックとして認識するようにテンポ提示方法を改良することが求められる.
本研究では,このボタン押下課題のシステムを改良し,遅延聴覚フィードバックが身体運動に与える影響をより明確に観察できるようにすることを目的とする.
