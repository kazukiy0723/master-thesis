%%%%%
%MATLABコードの表示
%%%%%
\chapter{遅延時間の測定に用いたプログラム}
\section{CPU負荷を100%に保つためのPythonファイル: cpu100.py}
\begin{verbatim}
    import multiprocessing
    import time
    import numpy as np
    import gc

    def cpu_100():
        while True:
            a = np.random.rand(5000, 5000)
            b = np.random.rand(5000, 5000)
            np.dot(a, b)

    if __name__ == "__main__":
        processes = []
        for _ in range(multiprocessing.cpu_count()):
            p = multiprocessing.Process(target=cpu_100)
            p.start()
            processes.append(p)

        # プロセスが終了しないように待機
        try:
            while True:
                time.sleep(1)
        except KeyboardInterrupt:
            for p in processes:
                p.terminate()


\end{verbatim}

\section{遅延時間測定のためのMATLABファイル: main.m}
\begin{verbatim}
    clear all

    filename = 'delayTime_test2.csv'; % 出力先ファイル
    fs = 48000; 
    nbits = 16; %  量子ビット数
    nChannel = 2; 
    buffer = 5000; % 標準偏差を算出するときのサンプル数
    length_record = 10; 

    recorder = audiorecorder(fs, nbits, nChannel);

    % 録音
    recordblocking(recorder, length_record); 

    % 配列の取得
    myRecording = getaudiodata(recorder);

    % 分割
    channel1 = myRecording(:, 1);
    if(nChannel == 2)
        channel2 = myRecording(:, 2);
    end

    % index2の計算
    analyzed_segment = channel2(1000:6000);
    % mean_val = mean(analyzed_segment);
    std_val = std(analyzed_segment);
    threshold2 = 2 * std_val;
    % threshold2 = 0.2;
    index2 = find(abs(channel2) > threshold2); 

    try
        % index2からbuffer点前までの信号を切り取る
        start = max(1, index2(1)-buffer);  
        data_to_analyze = channel1(start:index2(1)-1);

        % index1の計算
        mean_val1 = mean(data_to_analyze);
        std_val1 = std(data_to_analyze);
        threshold1 = 2 * std_val1;
        % 平均+2標準偏差を初めて超えた点をindex1
        index1 = find(abs(data_to_analyze) > threshold1, 1) + start - 1;

        
        time = (1/fs) * (0:length(channel1)-1) * 1000;
        Channel1_ExceedTime = time(index1);
        fprintf("Channel.1: %f\r\n", Channel1_ExceedTime);

        if(nChannel == 2)
            Channel2_ExceedTime = time(index2(1));
            fprintf("Channel.2: %f\r\n", Channel2_ExceedTime);
            delayTime = Channel2_ExceedTime - Channel1_ExceedTime;
            fprintf("Delay: %f\r\n", delayTime);

            % CSVファイルへの書き込み
            writematrix(delayTime, filename, 'WriteMode', 'append');

        end
    catch
        disp('エラー');
        return;
    end

    figure(1); 
    plot(time, channel1);
    hold on;
    plot(time, channel2);
    line([Channel1_ExceedTime, Channel1_ExceedTime], ylim, 'Color', 'g', 'LineStyle', '--', 'LineWidth', 2);
    line([Channel2_ExceedTime, Channel2_ExceedTime], ylim, 'Color', 'b', 'LineStyle', '--', 'LineWidth', 2);
    
    xlabel('Time [ms]', 'FontSize', 20);  
    ylabel('Amplitude', 'FontSize', 20);  

    % 凡例の設定
    legend('Channel 1','Channel 2', 'FontSize', 20);

    hold off;

\end{verbatim}