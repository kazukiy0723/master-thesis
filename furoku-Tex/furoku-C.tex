% %%%%%%%%%%%%%%%%%%%%%%%%%%%%%%%%%%%%%%%%%%%%%%%%%%%%%%%%%%%%%%%%%
% % 付録c
% %%%%%%%%%%%%%%%%%%%%%%%%%%%%%%%%%%%%%%%%%%%%%%%%%%%%%%%%%%%%%%%%%
\chapter{ボタン押し課題におけるアプリケーションのプログラム}
開発したボタン押し課題におけるアプリケーションを掲載する.
本プログラムは,Microsoft Visual Studio 2022でコンパイルできるソースファイル,ヘッダファイルおよびリソースファイルである.
関数の目的別にファイルを分けて作成している.

\begin{lstlisting}[caption=main.cpp]
	#include<windows.h>
#include<tchar.h>
#include<stdio.h>
#include<Xinput.h>
#include"resource.h"
#include"ginclude.h"
#include"asio.h"
#include"asiodrivers.h"

// 追加ファイル
#include<chrono>
#include<iostream>
#include<sstream>
#include<fstream>
#include<iomanip>
#include<time.h>
#include<string>
#include<ctime>
#include<vector>
#include<cmath>
#include"window.h"
#include"main.h"
#include"file.h"
#include"dialogbox.h"

// Visualスタイル有効化
#pragma comment(linker,"\"/manifestdependency:type='win32' \
name='Microsoft.Windows.Common-Controls' version='6.0.0.0' \
processorArchitecture='*' publicKeyToken='6595b64144ccf1df' language='*'\"")

#pragma comment(lib,"winmm.lib")

using namespace std;

LRESULT CALLBACK	windowfunc(HWND, UINT, WPARAM, LPARAM);
void				bufferswitch(long, ASIOBool);
ASIOTime* bufferswitchtimeinfo(ASIOTime*, long, ASIOBool);
void				sampleratedidchange(ASIOSampleRate);
long				asiomessage(long, long, void*, double*);

extern AsioDrivers* asioDrivers;
const TCHAR			szWinName[] = _T("delayclicksound2");
TCHAR				strASIOInfo[LENSTR], strWaveFileInfo[LENSTR], strBuffInfo[LENSTR];
ASIODriverInfo		DriverInfo;
ASIOBufferInfo		BufInfo[MAXNUMINCHS + MAXNUMOUTCHS];
ASIOChannelInfo		ChannelInfo[MAXNUMINCHS + MAXNUMOUTCHS];
ASIOCallbacks		CallBacks;
int					numInChannels, numOutChannels, idDevice, idChannelClick, idChannelSound;
bool				SupportASIOOutputReady;
HMMIO				hMmio;
void* tmpWaveData, * WaveData;
int					lenWaveData, idxWaveData, BitsPerSample;
bool				isPlaying, isLoadedDriver, isOpenedDevice, isCreatedBuffer;
ASIOSampleType		SampleType;
extern long inlatency, outlatency;
extern int lenBuffer;
bool Finished = false; // 実験終了フラグ

// 追加変数
extern HWND hStaticNowTime, hStaticTime, hEdit1, htaticNumberOfloops, hStaticNumberOfloops2;
HFONT hFont1, hFont2, hFont3, hFont4, hFont5;
RECT recthEdit1 = { 198, 98, 702, 402 };
int call_count, temp_count;
bool wavePlayed = false;
extern int NumberOfloops;
// 処理時間計測用
std::chrono::system_clock::time_point start, before_start;
extern int Num;
// 時間差を保存するためのベクタ
vector<string> MarginTime;
// ボタンの押下回数を保持する変数
int CountButtonClicked;
// 親ウィンドウのハンドルを保持
HWND hParentWindow;
// 結果の出力先のパスを保持する変数
string CSVFILENAME;
// 実験方法の選択
bool LABNormal;
int TempNumberOfLoops;
// 試行回数を保持する変数
int TempNumberOfTrials;
// ファイル名
static char FileNameCSV[MAX_PATH];
static char FileNameINI[MAX_PATH];

// タイムスタンプ
string TimeStampButtonClicked = _T("Nothing");

// dialogbox.cppと共有
extern int DelayTiming, DelayTime_ms;
int DelayTimeUnique;

// window.cppと共有
extern RECT rGroupIni;
extern RECT rGroupIni2;
//WinMain関数
int WINAPI WinMain(_In_ HINSTANCE hThisInst, _In_opt_ HINSTANCE hPrevInst, _In_ LPSTR lpszArgs, _In_ int nWinMode)
{
	HWND			hwnd;
	MSG				msg;
	WNDCLASSEX		wcl;
	HACCEL			haccel;

	//// 二重起動防止
	//HANDLE hMutex = CreateMutex(NULL, TRUE, _T("MyAppMutex"));
	//if (GetLastError() == ERROR_ALREADY_EXISTS) {
	//	MessageBox(NULL, _T("This Application is already running!"), _T("警告"), MB_OK |  MB_ICONWARNING);
	//	return 0;
	//}

	//ウィンドウクラスの定義
	wcl.cbSize = sizeof(WNDCLASSEX);				//WNDCLASSEX構造体のサイズ
	wcl.style = 0;									//ウィンドウクラススタイル
	wcl.lpfnWndProc = windowfunc;					//ウィンドウ関数
	wcl.cbClsExtra = 0;								//ウィンドウクラスのエキストラ
	wcl.cbWndExtra = 0;								//ウィンドウインスタンスのエキストラ
	wcl.hInstance = hThisInst;						//このプログラムのインスタンスへのハンドル
	wcl.hIcon = LoadIcon(NULL, IDI_APPLICATION);	//アイコンへのハンドル
	wcl.hCursor = LoadCursor(NULL, IDC_ARROW);		//カーソルへのハンドル
	wcl.hbrBackground = (HBRUSH)COLOR_WINDOW;		//背景ブラシへのハンドル
	wcl.lpszMenuName = MAKEINTRESOURCE(IDR_MENU1);	//メニュー
	wcl.lpszClassName = szWinName;					//ウィンドウクラス名
	wcl.hIconSm = LoadIcon(NULL, IDI_WINLOGO);		//スモールアイコンへのハンドル

	//ウィンドウクラスの登録
	if (!RegisterClassEx(&wcl)) {
		return(0);
	}

	//ウィンドウの生成
	hwnd = CreateWindow(
		szWinName,				//ウィンドウクラス名
		szWinName,				//ウィンドウ名
		WS_OVERLAPPEDWINDOW,	//ウィンドウスタイル
		CW_USEDEFAULT,			//x座標
		CW_USEDEFAULT,			//y座標
		CW_USEDEFAULT,			//幅
		CW_USEDEFAULT,			//高さ
		HWND_DESKTOP,			//親ウィンドウへのハンドル
		NULL,					//メニューへのハンドル
		hThisInst,				//このプログラムのインスタンスへのハンドル	
		NULL					//追加引数
	);

	// 親ウィンドウハンドルの記憶
	hParentWindow = hwnd;

	//ウィンドウの表示
	ShowWindow(hwnd, nWinMode);
	UpdateWindow(hwnd);

	//キーボードアクセラレータのロード
	haccel = LoadAccelerators(hThisInst, MAKEINTRESOURCE(IDR_ACCELERATOR1));

	//メッセージループの生成
	while (GetMessage(&msg, NULL, 0, 0)) {
		if (!TranslateAccelerator(hwnd, haccel, &msg)) {
			TranslateMessage(&msg);
			DispatchMessage(&msg);
		}
	}

	return((int)msg.wParam);
}

//ウィンドウ関数
LRESULT CALLBACK windowfunc(HWND hwnd, UINT message, WPARAM wparam, LPARAM lparam)
{
	ASIOError		asioresult;
	ASIOSampleRate	rate;
	char			devnames[MAXNUMDEVS][LENDEVNAME], * tmpnames[MAXNUMDEVS];
	TCHAR			str[LENSTR];
	long			numinchs, numoutchs, minlenbuffer, maxlenbuffer, preflenbuffer, granularity, tmplong;
	int				numdevs, idev, ich, ibuf, isample;
	bool			boolresult;
	double			tmpdouble;
	TCHAR			wavefilename[] = WAVEFILENAME;
	MMCKINFO		mmckinfoparent, mmckinfosubchunk;
	WAVEFORMATEX	wf;

	switch (message) {
	case WM_CREATE:
		//ウィンドウの初期化
		// ウィンドウの作成
		CreateControl(hwnd, wparam, lparam);
		// 実行ファイルのパス名の取得
		TCHAR lpFileName[MAX_PATH];
		GetModuleFileName(NULL, lpFileName, sizeof(lpFileName));
		// ウィンドウのタイトルを実行ファイルのパス名に変更
		SetWindowText(hwnd, lpFileName);
		//パラメータの初期化
		idDevice = -1;
		hMmio = NULL;
		tmpWaveData = NULL;
		WaveData = NULL;
		isPlaying = false;
		isLoadedDriver = false;
		isOpenedDevice = false;
		isCreatedBuffer = false;
		idxWaveData = 0;
		// 追加
		call_count = 1;
		temp_count = 0;
		NumberOfloops = 3;
		TempNumberOfLoops = 1;
		CountButtonClicked = 0;
		//LABNormal = true;
		TempNumberOfTrials = 34;
		//ASIOデバイス用メモリ領域の確保
		if (asioDrivers == NULL) {
			asioDrivers = new AsioDrivers();
		}

		//ASIOデバイス名の取得
		for (idev = 0; idev < MAXNUMDEVS; idev++) {
			tmpnames[idev] = devnames[idev];
		}
		numdevs = asioDrivers->getDriverNames(tmpnames, MAXNUMDEVS);
		if (numdevs == 0) {
			MessageBox(NULL, _T("asioDrivers->getDriverNames"), _T("Error"), MB_ICONSTOP | MB_OK);
		}

		//ASIOデバイスのドライバのロード
		boolresult = asioDrivers->loadDriver(devnames[IDDEV]);
		if (boolresult == false) {
			isLoadedDriver = false;
			_stprintf_s(str, LENSTR, _T("asioDrivers->loadDriver\nDevice Name: %s\n"), devnames[IDDEV]);
			MessageBox(NULL, str, _T("Error"), MB_ICONSTOP | MB_OK);
		}
		else {
			isLoadedDriver = true;
		}

		//ASIOデバイスのオープン
		memset(&DriverInfo, 0, sizeof(ASIODriverInfo));
		asioresult = ASIOInit(&DriverInfo);
		if (asioresult != ASE_OK) {
			isOpenedDevice = false;
			_stprintf_s(str, LENSTR, _T("ASIOInit\nDevice Name: %s\n"), devnames[IDDEV]);
			MessageBox(NULL, str, _T("Error"), MB_ICONSTOP | MB_OK);

			//ASIOデバイスのドライバのアンロード
			if (isLoadedDriver == true) {
				asioDrivers->removeCurrentDriver();
				isLoadedDriver = false;
			}
		}
		else {
			isOpenedDevice = true;
		}

		if (isOpenedDevice == false) {
			for (idev = 0; idev < numdevs; idev++) {
				//				for (idev = numdevs-1; idev >=0; idev--) {
				if (idev != IDDEV) {
					boolresult = asioDrivers->loadDriver(devnames[idev]);
					if (boolresult == false) {
						isLoadedDriver = false;
						_stprintf_s(str, LENSTR, _T("asioDrivers->loadDriver\nDevice Name: %s\n"), devnames[idev]);
						MessageBox(NULL, str, _T("Error"), MB_ICONSTOP | MB_OK);
					}
					else {
						isLoadedDriver = true;
					}

					//ASIOデバイスのオープン
					memset(&DriverInfo, 0, sizeof(ASIODriverInfo));
					asioresult = ASIOInit(&DriverInfo);
					if (asioresult != ASE_OK) {
						isOpenedDevice = false;
						_stprintf_s(str, LENSTR, _T("ASIOInit\nDevice Name: %s\n"), devnames[idev]);
						MessageBox(NULL, str, _T("Error"), MB_ICONSTOP | MB_OK);

						//ASIOデバイスのドライバのアンロード
						if (isLoadedDriver == true) {
							asioDrivers->removeCurrentDriver();
							isLoadedDriver = false;
						}
					}
					else {
						isOpenedDevice = true;
						break;
					}
				}
			}
		}

		//ASIOデバイスのサンプリング周波数のサポートの確認
		rate = FS;
		asioresult = ASIOCanSampleRate(rate);
		if (asioresult != ASE_OK) {
			MessageBox(NULL, _T("ASIOCanSampleRate"), _T("Error"), MB_ICONSTOP | MB_OK);
		}

		//ASIOデバイスのサンプリング周波数の設定
		asioresult = ASIOSetSampleRate(rate);
		if (asioresult != ASE_OK) {
			MessageBox(NULL, _T("ASIOSetSampleRate"), _T("Error"), MB_ICONSTOP | MB_OK);
		}

		//ASIOデバイスのチャンネル数の取得
		asioresult = ASIOGetChannels(&numinchs, &numoutchs);
		if (asioresult != ASE_OK) {
			MessageBox(NULL, _T("ASIOGetChannels"), _T("Error"), MB_ICONSTOP | MB_OK);
		}

		//ASIOデバイスの録音チャンネル数の設定
		if (numinchs <= MAXNUMINCHS) {
			numInChannels = numinchs;
		}
		else {
			numInChannels = MAXNUMINCHS;
		}

		//ASIOデバイスの再生用チャンネル数の設定
		if (numoutchs <= MAXNUMOUTCHS) {
			numOutChannels = numoutchs;
		}
		else {
			numOutChannels = MAXNUMOUTCHS;
		}
		if (CHCLICK <= numOutChannels) {
			idChannelClick = CHCLICK - 1;
		}
		else {
			idChannelClick = 0;
		}
		if (CHSOUND <= numOutChannels) {
			idChannelSound = CHSOUND - 1;
		}
		else {
			idChannelSound = 0;
		}

		//ASIOデバイスのバッファサイズの取得
		asioresult = ASIOGetBufferSize(&minlenbuffer, &maxlenbuffer, &preflenbuffer, &granularity);
		if (asioresult != ASE_OK) {
			MessageBox(NULL, _T("ASIOGetBufferSize"), _T("Error"), MB_ICONSTOP | MB_OK);
		}

		//ASIOデバイスのバッファサイズの設定
		lenBuffer = preflenbuffer;
		// ASIOバッファサイズが16ではない場合、メッセージボックスにて警告を出す
		if (lenBuffer != 16) {
			_stprintf_s(strBuffInfo, LENSTR,
			 _T("ASIOバッファサイズが「%d」に設定されています。\r\n一度ウィンドウを閉じてから「16」に設定しなおしてください。"), lenBuffer);
			MessageBox(NULL, strBuffInfo, _T("ASIOバッファサイズ"), MB_OK | MB_ICONWARNING);
		}

		//ASIOデバイスの録音用バッファの初期化
		for (ich = 0; ich < numInChannels; ich++) {
			memset(&(BufInfo[ich]), 0, sizeof(ASIOBufferInfo));
			BufInfo[ich].isInput = ASIOTrue;
			BufInfo[ich].channelNum = ich;
		}

		//ASIOデバイスの再生用バッファの初期化
		for (ich = 0; ich < numOutChannels; ich++) {
			memset(&(BufInfo[numInChannels + ich]), 0, sizeof(ASIOBufferInfo));
			BufInfo[numInChannels + ich].isInput = ASIOFalse;
			BufInfo[numInChannels + ich].channelNum = ich;
		}

		//ASIOデバイス用コールバック関数の設定
		CallBacks.bufferSwitch = &bufferswitch;
		CallBacks.bufferSwitchTimeInfo = &bufferswitchtimeinfo;
		CallBacks.sampleRateDidChange = &sampleratedidchange;
		CallBacks.asioMessage = &asiomessage;

		//ASIOデバイスのバッファの生成
		asioresult = ASIOCreateBuffers(BufInfo, numInChannels + numOutChannels, lenBuffer, &CallBacks);
		if (asioresult != ASE_OK) {
			isCreatedBuffer = false;
			MessageBox(NULL, _T("ASIOCreateBuffers"), _T("Error"), MB_ICONSTOP | MB_OK);
		}
		else {
			isCreatedBuffer = true;
		}

		//ASIOデバイスのチャンネル情報の取得
		for (ich = 0; ich < numInChannels + numOutChannels; ich++) {
			memset(&(ChannelInfo[ich]), 0, sizeof(ASIOChannelInfo));
			ChannelInfo[ich].isInput = BufInfo[ich].isInput;
			ChannelInfo[ich].channel = BufInfo[ich].channelNum;
			asioresult = ASIOGetChannelInfo(&(ChannelInfo[ich]));
			if (asioresult != ASE_OK) {
				MessageBox(NULL, _T("ASIOGetChannelInfo"), _T("Error"), MB_ICONSTOP | MB_OK);
			}
		}
		SampleType = ChannelInfo[0].type;
		switch (SampleType) {
		case ASIOSTInt16LSB:
			BitsPerSample = 16;
			break;
		case ASIOSTInt24LSB:
			BitsPerSample = 24;
			break;
		case ASIOSTInt32LSB:
			BitsPerSample = 32;
			break;
		default:
			break;
		}

		//ASIOデバイスのレイテンシの取得
		asioresult = ASIOGetLatencies(&inlatency, &outlatency);
		if (asioresult != ASE_OK) {
			MessageBox(NULL, _T("ASIOGetLatencies"), _T("Error"), MB_ICONSTOP | MB_OK);
		}

		//ASIOOutputReady関数のサポートの確認
		asioresult = ASIOOutputReady();
		if (asioresult == ASE_OK) {
			SupportASIOOutputReady = true;
		}
		else {
			SupportASIOOutputReady = false;
		}

		//ASIOデバイスの設定内容の確認
		ASIOGetSampleRate(&rate);
		_stprintf_s(strASIOInfo, LENSTR,
			_T("Device Name: %s\n")
			_T("Max. No. Input Channels: %d\n")
			_T("Max. No. Output Channels: %d\n")
			_T("Sampling Frequency: %f Hz\n")
			_T("Min. Buffer Size: %d points = %f ms\n")
			_T("Max. Buffer Size: %d points = %f ms\n")
			_T("Preferred Buffer Size: %d points = %f ms\n")
			_T("Allocated Buffer Size: %d points = %f ms\n")
			_T("Input Latency: %d points = %f ms\n")
			_T("Output Latency: %d points = %f ms\n")
			_T("Total Latency: %d points = %f ms\n")
			_T("Support \"ASIOOutputReady\": %s"),
			DriverInfo.name,
			numinchs,
			numoutchs,
			rate,
			minlenbuffer, (double)minlenbuffer * 1000.0 / rate,
			maxlenbuffer, (double)maxlenbuffer * 1000.0 / rate,
			preflenbuffer, (double)preflenbuffer * 1000.0 / rate,
			lenBuffer, (double)lenBuffer * 1000.0 / rate,
			inlatency, (double)inlatency * 1000.0 / rate,
			outlatency, (double)outlatency * 1000.0 / rate,
			inlatency + outlatency, ((double)inlatency + (double)outlatency) * 1000.0 / rate,
			SupportASIOOutputReady == true ? _T("Supported.") : _T("Not Supported."));
		//MessageBox(NULL, strASIOInfo, _T("ASIO"), MB_OK);

		// 遅延時間の設定用コンボボックスの作成とiniファイルの読み込み
		ReadIniFile(hwnd, lparam, NULL);
		// ループ回数の更新
		LateIniFunc(hwnd, wparam);

		//WAVEファイルのオープン
		if ((hMmio = mmioOpen(wavefilename, NULL, MMIO_READ)) == NULL) {
			//ファイルをオープンできなかった場合の処理
			_stprintf_s(str, LENSTR, _T("エラー: ファイルを開くことができません\nファイル名: %s"), wavefilename);
			MessageBox(hwnd, str, _T("ファイルを開く"), MB_ICONSTOP | MB_OK);
			return(0);
		}

		//ファイルのWAVEチャンクへ移動
		memset(&mmckinfoparent, 0, sizeof(MMCKINFO));
		mmckinfoparent.fccType = mmioFOURCC('W', 'A', 'V', 'E');
		if (mmioDescend(hMmio, &mmckinfoparent, NULL, MMIO_FINDRIFF) != MMSYSERR_NOERROR) {
			//チャンクを移動できなかった場合の処理
			_stprintf_s(str, LENSTR, _T("エラー: データを読み込むことができません\nファイル名: %s"), wavefilename);
			MessageBox(hwnd, str, _T("ファイルを開く"), MB_ICONSTOP | MB_OK);
			return(0);
		}

		//ファイルのfmtチャンクへ移動
		memset(&mmckinfosubchunk, 0, sizeof(MMCKINFO));
		mmckinfosubchunk.ckid = mmioFOURCC('f', 'm', 't', ' ');
		if (mmioDescend(hMmio, &mmckinfosubchunk, &mmckinfoparent, MMIO_FINDCHUNK) != MMSYSERR_NOERROR) {
			//チャンクを移動できなかった場合の処理
			_stprintf_s(str, LENSTR, _T("エラー: データを読み込むことができません\nファイル名: %s"), wavefilename);
			MessageBox(hwnd, str, _T("ファイルを開く"), MB_ICONSTOP | MB_OK);
			return(0);
		}

		//ファイルのデータ形式の読み込み
		memset(&wf, 0, sizeof(WAVEFORMATEX));
		if (mmioRead(hMmio, (HPSTR)&wf, mmckinfosubchunk.cksize) == -1) {
			//ファイルのデータ形式を読み込めなかった場合の処理
			_stprintf_s(str, LENSTR, _T("エラー: データを読み込むことができません\nファイル名: %s"), wavefilename);
			MessageBox(hwnd, str, _T("ファイルを開く"), MB_ICONSTOP | MB_OK);
			return(0);
		}

		//ファイルのデータ形式の確認
		if (wf.wFormatTag != WAVE_FORMAT_PCM) {
			//ファイルのデータ形式がPCMではなかった場合の処理
			_stprintf_s(str, LENSTR, 
			_T("エラー: PCM形式WAVEデータではないので読み込むことができません\nファイル名: %s"), wavefilename);
			MessageBox(hwnd, str, _T("ファイルを開く"), MB_ICONSTOP | MB_OK);
			return(0);
		}

		//ファイルのdataチャンクへ移動
		if (mmioAscend(hMmio, &mmckinfosubchunk, 0) != MMSYSERR_NOERROR) {
			//チャンクを移動できなかった場合の処理
			_stprintf_s(str, LENSTR, _T("エラー: データを読み込むことができません\nファイル名: %s"), wavefilename);
			MessageBox(hwnd, str, _T("ファイルを開く"), MB_ICONSTOP | MB_OK);
			return(0);
		}
		memset(&mmckinfosubchunk, 0, sizeof(MMCKINFO));
		mmckinfosubchunk.ckid = mmioFOURCC('d', 'a', 't', 'a');
		if (mmioDescend(hMmio, &mmckinfosubchunk, &mmckinfoparent, MMIO_FINDCHUNK) != MMSYSERR_NOERROR) {
			//チャンクを移動できなかった場合の処理
			_stprintf_s(str, LENSTR, _T("エラー: データを読み込むことができません\nファイル名: %s"), wavefilename);
			MessageBox(hwnd, str, _T("ファイルを開く"), MB_ICONSTOP | MB_OK);
			return(0);
		}
		lenWaveData = mmckinfosubchunk.cksize / wf.nBlockAlign;

		//メッセージボックスへのテキストの出力
		_stprintf_s(strWaveFileInfo, LENSTR,
			_T("File Name: %s\n")
			_T("Sampling Frequency: %d Hz\n")
			_T("Bits per Sample: %d bits\n")
			_T("Number of Channels: %d channels\n")
			_T("Length of Data: %d samples"),
			wavefilename, wf.nSamplesPerSec, wf.wBitsPerSample, wf.nChannels, lenWaveData);
		//MessageBox(NULL, strWaveFileInfo, _T("Read a Wave File"), MB_OK);

		//WAVEデータ用バッファの生成
		tmpWaveData = (void*)malloc(mmckinfosubchunk.cksize);
		WaveData = (void*)malloc(lenWaveData * (BitsPerSample / 8));
		if (tmpWaveData == NULL || WaveData == NULL) {
			//バッファを生成できなかった場合の処理
			MessageBox(hwnd, _T("エラー: メモリーが足りません"), _T("ファイルを開く"), MB_ICONSTOP | MB_OK);
			return(0);
		}

		//WAVEデータの読み込み
		if (mmioRead(hMmio, (HPSTR)tmpWaveData, mmckinfosubchunk.cksize) <= 0) {
			//データを読み込めなかった場合の処理
			_stprintf_s(str, LENSTR, _T("エラー: データを読み込むことができません\nファイル名: %s"), wavefilename);
			MessageBox(hwnd, str, _T("ファイルを開く"), MB_ICONSTOP | MB_OK);
			return(0);
		}

		//WAVEファイルのクローズ
		if (hMmio != NULL) {
			mmioClose(hMmio, 0);
			hMmio = NULL;
		}

		//WAVEデータの変換
		for (isample = 0; isample < lenWaveData; isample++) {
			tmpdouble = 0.0;
			for (ich = 0; ich < wf.nChannels; ich++) {
				//WAVEデータのshort int型への変換
				switch (wf.wBitsPerSample) {
				case 8:
					tmplong = ((long)((unsigned char*)tmpWaveData)[isample * wf.nChannels + ich] - 0x80L) * 0x100L;
					break;
				case 16:
					tmplong = (long)((short int*)tmpWaveData)[isample * wf.nChannels + ich];
					break;
				}
				if (wf.nChannels > 1) {
					tmpdouble += (double)tmplong;
				}
			}
			if (wf.nChannels > 1) {
				tmplong = (long)(tmpdouble / (double)wf.nChannels);
			}

			//WAVEデータのshort int型からの変換
			switch (SampleType) {
			case ASIOSTInt16LSB:
				((char*)WaveData)[isample * 2] = (char)tmplong;
				tmplong >>= 8;
				((char*)WaveData)[isample * 2 + 1] = (char)tmplong;
				break;
			case ASIOSTInt24LSB:
				tmplong *= 0x100L;
				((char*)WaveData)[isample * 3] = (char)tmplong;
				tmplong >>= 8;
				((char*)WaveData)[isample * 3 + 1] = (char)tmplong;
				tmplong >>= 8;
				((char*)WaveData)[isample * 3 + 2] = (char)tmplong;
				break;
			case ASIOSTInt32LSB:
				tmplong *= 0x10000L;
				((char*)WaveData)[isample * 4] = (char)tmplong;
				tmplong >>= 8;
				((char*)WaveData)[isample * 4 + 1] = (char)tmplong;
				tmplong >>= 8;
				((char*)WaveData)[isample * 4 + 2] = (char)tmplong;
				tmplong >>= 8;
				((char*)WaveData)[isample * 4 + 3] = (char)tmplong;
				break;
			default:
				break;
			}
		}

		//WAVEデータ用バッファの解放
		if (tmpWaveData != NULL) {
			free(tmpWaveData);
			tmpWaveData = NULL;
		}

		// フォントの初期設定
		OnFont(hwnd);

		//ASIOデバイスの再生用バッファの初期化
		for (ich = 0; ich < numOutChannels; ich++) {
			for (ibuf = 0; ibuf < NUMASIOBUFFERS; ibuf++) {
				switch (SampleType) {
				case ASIOSTInt16LSB:
					memset(BufInfo[numInChannels + ich].buffers[ibuf], 0, lenBuffer * 2);
					break;
				case ASIOSTInt24LSB:
					memset(BufInfo[numInChannels + ich].buffers[ibuf], 0, lenBuffer * 3);
					break;
				case ASIOSTInt32LSB:
					memset(BufInfo[numInChannels + ich].buffers[ibuf], 0, lenBuffer * 4);
					break;
				default:
					break;
				}
			}
		}

		//ASIOデバイスの開始
		asioresult = ASIOStart();
		if (asioresult != ASE_OK) {
			MessageBox(NULL, _T("ASIOStart"), _T("Error"), MB_ICONSTOP | MB_OK);
		}
		break;

	case WM_COMMAND:
		OnCommand(hwnd, wparam, lparam);
		break;
	case WM_CLOSE:
		//プログラムの終了
		if (IDYES == MessageBox(hwnd, _T("終了しますか?"), _T("終了確認"), MB_YESNO)) {
			DestroyWindow(hwnd);
		}
		break;
	case WM_PAINT:
	{
		PAINTSTRUCT ps;
		HDC hdc = BeginPaint(hwnd, &ps);
		OnPaint(hwnd, hdc);

		EndPaint(hwnd, &ps);
	}
	break;

	case WM_DESTROY:

		// 後始末
		DeleteObject(hFont1);
		DeleteObject(hFont2);
		DeleteObject(hFont3);
		DeleteObject(hFont4);
		DeleteObject(hFont5);

		//ASIOデバイスの停止
		asioresult = ASIOStop();
		if (asioresult != ASE_OK) {
			MessageBox(NULL, _T("ASIOStop"), _T("Error"), MB_ICONSTOP | MB_OK);
		}

		//ASIOデバイスのバッファの解放
		asioresult = ASIODisposeBuffers();
		if (asioresult != ASE_OK) {
			MessageBox(NULL, _T("ASIODisposeBuffers"), _T("Error"), MB_ICONSTOP | MB_OK);
		}

		//ASIOデバイスのクローズ
		if (isOpenedDevice == true) {
			asioresult = ASIOExit();
			if (asioresult != ASE_OK) {
				MessageBox(NULL, _T("ASIOExit"), _T("Error"), MB_ICONSTOP | MB_OK);
			}
			isOpenedDevice = false;
		}

		//ASIOデバイスのドライバのアンロード
		if (isLoadedDriver == true) {
			asioDrivers->removeCurrentDriver();
			isLoadedDriver = false;
		}

		//ASIOデバイス用メモリ領域の解放
		if (asioDrivers != NULL) {
			delete asioDrivers;
		}

		//WAVEデータ用バッファの解放
		if (tmpWaveData != NULL) {
			free(tmpWaveData);
			tmpWaveData = NULL;
		}
		if (WaveData != NULL) {
			free(WaveData);
			WaveData = NULL;
		}

		//WAVEファイルのクローズ
		if (hMmio != NULL) {
			mmioClose(hMmio, 0);
			hMmio = NULL;
		}

		PostQuitMessage(0);
		break;

	default:
		return(DefWindowProc(hwnd, message, wparam, lparam));
	}

	return(0);
}



//ASIOデバイスのバッファ処理用コールバック関数
void bufferswitch(long index, ASIOBool processNow) {
	ASIOError	asioresult;
	int			tmplenbuffer, isample, position;
	double	tmpdouble;
	long		tmplong;
	

	//録音データ(クリック信号)の変換
	for (isample = 0; isample < lenBuffer; isample++) {
		switch (SampleType) {
		case ASIOSTInt16LSB:
			tmplong = 0L;
			tmplong += ((unsigned char*)(BufInfo[idChannelClick].buffers[index]))[isample * 2 + 1];
			tmplong <<= 8;
			tmplong += ((unsigned char*)(BufInfo[idChannelClick].buffers[index]))[isample * 2];
			tmpdouble = (double)tmplong / (double)0x8000L;
			break;
		case ASIOSTInt24LSB:
			tmplong = 0L;
			tmplong += ((unsigned char*)(BufInfo[idChannelClick].buffers[index]))[isample * 3 + 2];
			tmplong <<= 8;
			tmplong += ((unsigned char*)(BufInfo[idChannelClick].buffers[index]))[isample * 3 + 1];
			tmplong <<= 8;
			tmplong += ((unsigned char*)(BufInfo[idChannelClick].buffers[index]))[isample * 3];
			tmpdouble = (double)tmplong / (double)0x800000L;
			break;
		case ASIOSTInt32LSB:
			tmplong = 0L;
			tmplong += ((unsigned char*)(BufInfo[idChannelClick].buffers[index]))[isample * 4 + 3];
			tmplong <<= 8;
			tmplong += ((unsigned char*)(BufInfo[idChannelClick].buffers[index]))[isample * 4 + 2];
			tmplong <<= 8;
			tmplong += ((unsigned char*)(BufInfo[idChannelClick].buffers[index]))[isample * 4 + 1];
			tmplong <<= 8;
			tmplong += ((unsigned char*)(BufInfo[idChannelClick].buffers[index]))[isample * 4];
			tmpdouble = (double)tmplong / (double)0x80000000UL;
			break;
		default:
			tmpdouble = 0.0;
			break;
		}

		if (tmpdouble >= THRESH || -tmpdouble >= THRESH) {
			if (isPlaying == false || idxWaveData + isample > (FS * LENCHATTERING) / 1000) {
				// パラメータの設定
				isPlaying = true;
				idxWaveData = 0;
				position = isample;
				call_count = 1;
				// 前回のボタン押下時刻の取得
				if (CountButtonClicked)  before_start = start;
				// 現在時刻の取得
				start = std::chrono::system_clock::now();
				// エディットボックスにボタンの押下時刻を表示
				SendNowTimeToEdit(hEdit1, getMilliTime(start));
				if (CountButtonClicked) {
					// 前回のボタン押下時刻との時間差を計算し、表示
					auto time = start - before_start;
					SendMarginTimeToEdit(time);
					// ベクター配列に結果を保存
					GetMarginTime(time);
					// テキストの最後にカーソルを移動
					ScrollToBottom(hEdit1);
				}
				if (!LABNormal) {
					/////////////////////////
					// 実験方法が変則の場合
					// ///////////////////////
					/*ここにクリック音の再生タイミングをずらすためのプログラムを書く。*/
					if (!((CountButtonClicked+1) % DelayTiming)) {
						// 再生タイミングをずらす(1回目以外)
						CalcLateNumberOfloops(hParentWindow, &DelayTimeUnique, DelayTime_ms, lenBuffer, FS, inlatency, outlatency);
						NumberOfloops = NumberOfloops + DelayTimeUnique;
					}
					else if(NumberOfloops != TempNumberOfLoops) {
						NumberOfloops = TempNumberOfLoops;
					}
				}
				// 実験終了のためのメッセージボックスを出力
				if (CountButtonClicked == TempNumberOfTrials) {
					string TempNumberString = _T("指定回数に達しました。");
					MessageBox(NULL, TempNumberString.c_str(), _T("終了の合図"), MB_OK);
					SetStddev2Window(MarginTime);                                                   // 標準偏差を計算し、ウィンドウに表示
					
					// 無効化
					EnableWindow(GetDlgItem(hParentWindow, ID_LATESETTING), FALSE);
					EnableWindow(GetDlgItem(hParentWindow, ID_LATESETTING_2), FALSE);
					EnableWindow(GetDlgItem(hParentWindow, ID_LATEINI_2), FALSE);
					EnableWindow(GetDlgItem(hParentWindow, ID_EDIT_LATEDATA_TIMING), FALSE);
					EnableWindow(GetDlgItem(hParentWindow, ID_LATEINI), FALSE);
				}
				// ボタンの押下回数を更新
				SendCountButtonClicked(GetDlgItem(hParentWindow, ID_STATIC_COUNTBUTTONCLICKED_2), CountButtonClicked);
				Finished = true; // 音が鳴らないようにするための変数
				break;
			}
		}
	}
	if (isample == lenBuffer) {
		position = 0;
	}

	//ASIOデバイスのクリック音再生用バッファのクリア
	switch (SampleType) {
	case ASIOSTInt16LSB:
		memset(BufInfo[numInChannels + idChannelClick].buffers[index], 0, lenBuffer * 2);
		break;
	case ASIOSTInt24LSB:
		memset(BufInfo[numInChannels + idChannelClick].buffers[index], 0, lenBuffer * 3);
		break;
	case ASIOSTInt32LSB:
		memset(BufInfo[numInChannels + idChannelClick].buffers[index], 0, lenBuffer * 4);
		break;
	default:
		break;
	}

	if (isPlaying == true) {
		if (lenBuffer - position > lenWaveData - idxWaveData) {
			tmplenbuffer = lenWaveData - idxWaveData;
		}
		else {
			tmplenbuffer = lenBuffer - position;
		}
		// ボタンが押されてから(NumberOfloops)回目のbufferswitch呼び出し時にWaveデータをコピー
		if (call_count % NumberOfloops != 0) {
			// 録音データ(音声信号)のコピー(出力バッファー2にコピー)
			switch (SampleType) {
			case ASIOSTInt16LSB:
				memcpy_s(BufInfo[numInChannels + idChannelClick].buffers[index], lenBuffer * 2, BufInfo[idChannelSound].buffers[index], lenBuffer * 2);
				break;
			case ASIOSTInt24LSB:
				memcpy_s(BufInfo[numInChannels + idChannelClick].buffers[index], lenBuffer * 3, BufInfo[idChannelSound].buffers[index], lenBuffer * 3);
				break;
			case ASIOSTInt32LSB:
				memcpy_s(BufInfo[numInChannels + idChannelClick].buffers[index], lenBuffer * 4, BufInfo[idChannelSound].buffers[index], lenBuffer * 4);
				break;
			default:
				break;
			}
			call_count++;
		}
		else {
			//WAVEデータのクリック音再生用バッファへのコピー(出力バッファー1と2にコピー)
			switch (SampleType) {
			case ASIOSTInt16LSB:
				memcpy_s((void*)&(((char*)(BufInfo[numInChannels + idChannelClick].buffers[index]))[position * 2]), tmplenbuffer * 2, (void*)&(((char*)WaveData)[idxWaveData * 2]), tmplenbuffer * 2);
				memcpy_s((void*)&(((char*)(BufInfo[numInChannels + idChannelSound].buffers[index]))[position * 2]), tmplenbuffer * 2, (void*)&(((char*)WaveData)[idxWaveData * 2]), tmplenbuffer * 2);
				break;
			case ASIOSTInt24LSB:
				memcpy_s((void*)&(((char*)(BufInfo[numInChannels + idChannelClick].buffers[index]))[position * 3]), tmplenbuffer * 3, (void*)&(((char*)WaveData)[idxWaveData * 3]), tmplenbuffer * 3);
				memcpy_s((void*)&(((char*)(BufInfo[numInChannels + idChannelSound].buffers[index]))[position * 3]), tmplenbuffer * 3, (void*)&(((char*)WaveData)[idxWaveData * 3]), tmplenbuffer * 3);
				break;
			case ASIOSTInt32LSB:
				memcpy_s((void*)&(((char*)(BufInfo[numInChannels + idChannelClick].buffers[index]))[position * 4]), tmplenbuffer * 4, (void*)&(((char*)WaveData)[idxWaveData * 4]), tmplenbuffer * 4);
				memcpy_s((void*)&(((char*)(BufInfo[numInChannels + idChannelSound].buffers[index]))[position * 4]), tmplenbuffer * 4, (void*)&(((char*)WaveData)[idxWaveData * 4]), tmplenbuffer * 4);
				break;
			default:
				break;
			}
			idxWaveData += tmplenbuffer;
			if (idxWaveData >= lenWaveData) {
				/////////////////////
				//パラメータの設定
				////////////////////
				isPlaying = false;
				idxWaveData = 0;
				call_count = 1;
			}
			
		}
	}
		//録音データ(音声信号)のコピー(出力バッファー1にコピー)
		switch (SampleType) {
		case ASIOSTInt16LSB:
			memcpy_s(BufInfo[numInChannels + idChannelSound].buffers[index], lenBuffer * 2, BufInfo[idChannelSound].buffers[index], lenBuffer * 2);
			break;
		case ASIOSTInt24LSB:
			memcpy_s(BufInfo[numInChannels + idChannelSound].buffers[index], lenBuffer * 3, BufInfo[idChannelSound].buffers[index], lenBuffer * 3);
			break;
		case ASIOSTInt32LSB:
			memcpy_s(BufInfo[numInChannels + idChannelSound].buffers[index], lenBuffer * 4, BufInfo[idChannelSound].buffers[index], lenBuffer * 4);
			break;
		default:
			break;
		}
	

	//再生用バッファの準備完了の通知
	if (SupportASIOOutputReady == true) {
		asioresult = ASIOOutputReady();
		if (asioresult != ASE_OK) {
			MessageBox(NULL, _T("ASIOOutputReady"), _T("Error"), MB_ICONSTOP | MB_OK);
		}
	}
}

// 遅延時間[ms]を表示
bool SendMarginTimeToEdit(std::chrono::system_clock::duration time) {

	auto msec = std::chrono::duration_cast<std::chrono::milliseconds>(time).count();
	std::string string_msec = std::to_string(msec);
	std::string b = _T("時間差[ms]: ");
	string_msec = b + string_msec;
	const char* str_msec = string_msec.c_str();
	SendNowTimeToEdit(hEdit1, str_msec);
	

	return true;
}

// 時間差を配列に代入していくための関数
bool GetMarginTime(std::chrono::system_clock::duration time) {

	string stringTime;

	auto msec = std::chrono::duration_cast<std::chrono::milliseconds>(time).count();
	
	stringTime = to_string(msec);
	//else stringTime = _T("false"); 

	MarginTime.push_back(stringTime);

	return true;
}

// 得られた時間差をcsvファイルに書き込むための関数
bool WriteToCSV(HWND hwnd, vector<string>& MarginTime, const string& filename) {
	char Temp[MAX_PATH];
	_stprintf_s(Temp, MAX_PATH,
		_T("以下のファイルに書き込みました。\r\n%s"),
		filename.c_str());

	// 出力ファイルストリームオブジェクトを作成
	// ファイルが存在しない場合は新たにファイルを作成、存在する場合はファイルの末尾に書き込み
		ofstream file(filename, ios::app);
		// カンマ区切りでストリームに送る
		for (size_t i = 0; i < MarginTime.size(); ++i) {
			file << MarginTime[i];
			if (i != MarginTime.size() - 1) {
				file << ",";
			}
		}
		file << "\n";
		file.close();

		// ベクターの中身の削除
		MarginTime.clear();
		// 書き込み確認
		MessageBox(hwnd, Temp, _T("結果の出力先の確認"), MB_OK);
		// 押下回数の更新
		CountButtonClicked = 0;
		SetWindowText(GetDlgItem(hwnd, ID_STATIC_COUNTBUTTONCLICKED_2), _T("0"));
		// エディットボックスの表示内容をクリア
		SetWindowText(hEdit1, _T(""));

	return true;
}

// ベクターの中身を削除
bool ClearMarginTime(HWND hwnd, vector<string>& MarginTime) {

	// ベクターの中身の削除
	if (!MarginTime.empty()) {
		MarginTime.clear();
		// ボタンの押下回数の更新
		CountButtonClicked = 0;
		SetWindowText(GetDlgItem(hwnd, ID_STATIC_COUNTBUTTONCLICKED_2), _T("0"));
		// エディットボックスの表示内容をクリア
		SetWindowText(hEdit1, _T(""));
	}
	else
	{
		MessageBox(hwnd, _T("配列には何も入ってません。"), NULL, MB_OK);
	}
	return true;
}

// ボタンの押下回数の更新
bool SendCountButtonClicked(HWND hStatic, int& CountButtonClicked) {
	// ボタンの押下回数の送信
	CountButtonClicked++;
	string CountString = to_string(CountButtonClicked);
	SetWindowText(hStatic, CountString.c_str());
	return true;
}

// エディットボックスに時刻を表示
bool SendNowTimeToEdit(HWND hwndEdit, const char* timeStr) {

	// エディットボックス内の現在の文字列を取得
	int length = GetWindowTextLength(hwndEdit);
	char* existingText = new char[length + 1];
	GetWindowText(hwndEdit, existingText, length + 1);

	char* newText = new char[length + strlen(timeStr) + 3];
	sprintf(newText, "%s\r\n%s", existingText, timeStr);

	// 新しくエディットボックスに文字をセット
	SetWindowText(hwndEdit, newText);

	// 後始末
	delete[] existingText;
	delete[] newText;

	return true;
}

// 現在時刻をミリ秒まで取得(同時に呼び出されないようにする)
const char* getMilliTime(std::chrono::system_clock::time_point now) {
	// 現在の時刻を取得
	//auto now = std::chrono::system_clock::now();

	// エポックからの経過時間を秒とミリ秒に分割
	auto seconds = std::chrono::time_point_cast<std::chrono::seconds>(now);
	auto milliseconds = std::chrono::duration_cast<std::chrono::milliseconds>(now - seconds);

	// std::time_tに変換
	std::time_t tt = std::chrono::system_clock::to_time_t(seconds);

	// 時刻をローカルタイムに変換
	std::tm* tm = std::localtime(&tt); // この関数はスレッドセーフではない

	// 指定された形式で出力
	std::ostringstream oss, oss2;
	oss << std::put_time(tm, "%H:%M:%S") << '.'
		<< std::setfill('0') << std::setw(3) << milliseconds.count();

	// ファイル出力用
	oss2 << put_time(tm, "%m-%d");

	// 結果をstatic std::stringに変換
	static string str, str2, a;
	
	str = oss.str();
	str2 = oss2.str();

	// グローバル変数に時刻を記憶(csvファイルに書き込む際に必要)
	TimeStampButtonClicked = str2;
	a = _T("ボタン押下時刻: ");
	// 結果のconst char*を返す
	str = a + str;
	return str.c_str();
}




//ASIOデバイスのバッファ処理時のタイムスタンプ用コールバック関数
ASIOTime* bufferswitchtimeinfo(ASIOTime* params, long doubleBufferIndex, ASIOBool directProcess)
{
	
	return(0L);
}

//ASIOデバイスのサンプリング周波数変更検出用コールバック関数
void sampleratedidchange(ASIOSampleRate sRate)
{
	return;
}

//ASIOデバイスからのメッセージの処理用コールバック関数
long asiomessage(long selector, long value, void* message, double* opt)
{
	return(0L);
}



// NumberOfloopsの表示
bool DispNumberOfloops() {

	return true;
}
// フォントの設定
bool OnFont(HWND hwnd) {

	hFont1 = CreateFont(25, 0, 0, 0, FW_MEDIUM, false, false, false, SHIFTJIS_CHARSET, OUT_DEFAULT_PRECIS,
		CLIP_DEFAULT_PRECIS, DEFAULT_QUALITY, (VARIABLE_PITCH | FF_DONTCARE), _T("メイリオ"));

	hFont2 = CreateFont(22, 0, 0, 0, FW_MEDIUM, false, false, false, SHIFTJIS_CHARSET, OUT_DEFAULT_PRECIS,
		CLIP_DEFAULT_PRECIS, DEFAULT_QUALITY, (VARIABLE_PITCH | FF_DONTCARE), _T("メイリオ"));

	hFont3 = CreateFont(45, 0, 0, 0, FW_BOLD, false, false, false, SHIFTJIS_CHARSET, OUT_DEFAULT_PRECIS,
		CLIP_DEFAULT_PRECIS, DEFAULT_QUALITY, (VARIABLE_PITCH | FF_DONTCARE), _T("メイリオ"));

	hFont4 = CreateFont(18, 0, 0, 0, FW_MEDIUM, false, false, false, SHIFTJIS_CHARSET, OUT_DEFAULT_PRECIS,
		CLIP_DEFAULT_PRECIS, DEFAULT_QUALITY, (VARIABLE_PITCH | FF_DONTCARE), _T("メイリオ"));

	hFont5 = CreateFont(30, 0, 0, 0, FW_BOLD, false, false, false, SHIFTJIS_CHARSET, OUT_DEFAULT_PRECIS,
		CLIP_DEFAULT_PRECIS, DEFAULT_QUALITY, (VARIABLE_PITCH | FF_DONTCARE), _T("メイリオ"));

	SendMessage(hEdit1, WM_SETFONT, (WPARAM)hFont1, MAKELPARAM(false, 0));
	SendMessage(hStaticTime, WM_SETFONT, (WPARAM)hFont1, MAKELPARAM(false, 0));
	// ステティックコントロール
	SendMessage(GetDlgItem(hwnd, ID_STATIC_NUMBERDELAY1), WM_SETFONT, (WPARAM)hFont2, MAKELPARAM(false, 0));
	SendMessage(GetDlgItem(hwnd, ID_STATIC_NUMBERDELAY2), WM_SETFONT, (WPARAM)hFont2, MAKELPARAM(false, 0));
	SendMessage(GetDlgItem(hwnd, ID_STATIC_COUNTBUTTONCLICKED), WM_SETFONT, (WPARAM)hFont2, MAKELPARAM(false, 0));
	SendMessage(GetDlgItem(hwnd, ID_STATIC_COUNTBUTTONCLICKED_2), WM_SETFONT, (WPARAM)hFont3, MAKELPARAM(false, 0));
	SendMessage(GetDlgItem(hwnd, ID_STATIC_NUMBERDELAY1), WM_SETFONT, (WPARAM)hFont2, MAKELPARAM(false, 0));
	SendMessage(GetDlgItem(hwnd, ID_STATIC_NUMBERDELAY2), WM_SETFONT, (WPARAM)hFont2, MAKELPARAM(false, 0));
	SendMessage(GetDlgItem(hwnd, ID_GROUP_SETTING), WM_SETFONT, (WPARAM)hFont2, MAKELPARAM(false, 0));
	SendMessage(GetDlgItem(hwnd, ID_STATIC_BPM), WM_SETFONT, (WPARAM)hFont2, MAKELPARAM(false, 0));
	SendMessage(GetDlgItem(hwnd, ID_STATIC_BPMNUM), WM_SETFONT, (WPARAM)hFont2, MAKELPARAM(false, 0));
	SendMessage(GetDlgItem(hwnd, ID_STATIC_TRIALS), WM_SETFONT, (WPARAM)hFont2, MAKELPARAM(false, 0));
	SendMessage(GetDlgItem(hwnd, ID_STATIC_TRIALS_NUM), WM_SETFONT, (WPARAM)hFont2, MAKELPARAM(false, 0));
	SendMessage(GetDlgItem(hwnd, ID_STATIC_LAB), WM_SETFONT, (WPARAM)hFont2, MAKELPARAM(false, 0));
	SendMessage(GetDlgItem(hwnd, ID_STATIC_NORMALORIRREG), WM_SETFONT, (WPARAM)hFont2, MAKELPARAM(false, 0));
	SendMessage(GetDlgItem(hwnd, ID_GROUP_LATEDATA), WM_SETFONT, (WPARAM)hFont2, MAKELPARAM(false, 0));
	SendMessage(GetDlgItem(hwnd, ID_GROUP_LATEDATA_2), WM_SETFONT, (WPARAM)hFont2, MAKELPARAM(false, 0));
	SendMessage(GetDlgItem(hwnd, ID_STATIC_FILENAME), WM_SETFONT, (WPARAM)hFont2, MAKELPARAM(false, 0));
	SendMessage(GetDlgItem(hwnd, ID_STATIC_OLD), WM_SETFONT, (WPARAM)hFont2, MAKELPARAM(false, 0));
	SendMessage(GetDlgItem(hwnd, ID_STATIC_NAME), WM_SETFONT, (WPARAM)hFont2, MAKELPARAM(false, 0));
	SendMessage(GetDlgItem(hwnd, ID_GROUP_SUBJECTINFO), WM_SETFONT, (WPARAM)hFont2, MAKELPARAM(false, 0));
	SendMessage(GetDlgItem(hwnd, ID_EDIT_OLD), WM_SETFONT, (WPARAM)hFont2, MAKELPARAM(false, 0));
	SendMessage(GetDlgItem(hwnd, ID_EDIT_NAME), WM_SETFONT, (WPARAM)hFont2, MAKELPARAM(false, 0));
	SendMessage(GetDlgItem(hwnd, ID_STATIC_COMBO_MS), WM_SETFONT, (WPARAM)hFont2, MAKELPARAM(false, 0));
	SendMessage(GetDlgItem(hwnd, ID_STATIC_COMBO_MS_2), WM_SETFONT, (WPARAM)hFont2, MAKELPARAM(false, 0));
	SendMessage(GetDlgItem(hwnd, ID_STATIC_LATEDATATIMING), WM_SETFONT, (WPARAM)hFont2, MAKELPARAM(false, 0));
	SendMessage(GetDlgItem(hwnd, ID_STATIC_STDDEV), WM_SETFONT, (WPARAM)hFont2, MAKELPARAM(false, 0));

	// コンボボックス
	SendMessage(GetDlgItem(hwnd, ID_LATEINI), WM_SETFONT, (WPARAM)hFont1, MAKELPARAM(false, 0));
	SendMessage(GetDlgItem(hwnd, ID_LATEINI_2), WM_SETFONT, (WPARAM)hFont4, MAKELPARAM(false, 0));
	SendMessage(GetDlgItem(hwnd, ID_LATESETTING), WM_SETFONT, (WPARAM)hFont1, MAKELPARAM(false, 0));
	SendMessage(GetDlgItem(hwnd, ID_LATESETTING_2), WM_SETFONT, (WPARAM)hFont1, MAKELPARAM(false, 0));
	SendMessage(GetDlgItem(hwnd, ID_STATIC_LATESETTING_2), WM_SETFONT, (WPARAM)hFont2, MAKELPARAM(false, 0));

	// プッシュボタン
	SendMessage(GetDlgItem(hwnd, ID_BUTTON_CSV), WM_SETFONT, (WPARAM)hFont5
		, MAKELPARAM(false, 0));
	
	// エディットボックス
	SendMessage(GetDlgItem(hwnd, ID_EDIT_LATEDATA_TIMING), WM_SETFONT, (WPARAM)hFont2, MAKELPARAM(false, 0));
	SendMessage(GetDlgItem(hwnd, ID_EDIT_FILEPATH), WM_SETFONT, (WPARAM)hFont2, MAKELPARAM(false, 0));

	return true;
}



// コンボボックスにフォーカスが来た時
bool LateIniFunc(HWND hWnd, WPARAM wParam) {

		//コンボボックスで現在選択されている項目のインデックスを取得
		Num = GetNowComboStr(hWnd, ID_LATESETTING);
		// NumberOfloopsを計算
		CalcLateNumberOfloops(hWnd, &NumberOfloops, Num, lenBuffer, FS, inlatency, outlatency);
		// NumberOfloopsをstd::string型に変換
		string stringnum = to_string(NumberOfloops);
		// 画面上に表示
		SetWindowText(hStaticNumberOfloops2, stringnum.c_str());
		// グローバル変数にNumberOfLoopsの値を記憶させる
		TempNumberOfLoops = NumberOfloops;

	return true;
}

// コンボボックス(変則var)の項目が変更された時
bool LateIniFunc_2(HWND hWnd, WPARAM wParam) {

	//コンボボックスで現在選択されている項目のインデックスを取得
	int LocalDelayTIme = GetNowComboStr(hWnd, ID_LATESETTING_2);
	// 遅延のタイミング
	char* lpstringDelayTiming = new char[50];
	GetWindowText(GetDlgItem(hWnd, ID_EDIT_LATEDATA_TIMING), lpstringDelayTiming, 50);
	DelayTiming = atoi(lpstringDelayTiming);

	DelayTime_ms = LocalDelayTIme;

	delete[] lpstringDelayTiming;

	return true;
}

// コンボボックス1で選択されたキー名に対応した遅延時間をコンボボックスに詰める。
bool SendLate2Combo(HWND hwnd, WPARAM wParam, char* IniFilepath) {
		//コンボボックスで現在選択されている項目のインデックスを取得
		int intCurrentIndex = SendMessage(GetDlgItem(hwnd, (int)ID_LATEINI), CB_GETCURSEL, 0, 0);

		// 現在選択されている項目のの文字列の長さを取得
		int intTxtLen = SendMessage(GetDlgItem(hwnd, (int)ID_LATEINI), CB_GETLBTEXTLEN, intCurrentIndex, 0);

		if (intTxtLen != CB_ERR){
			char* pszBuf = new char[intTxtLen + 1];
			if (SendMessage(GetDlgItem(hwnd, ID_LATEINI), CB_GETLBTEXT, intCurrentIndex, (LPARAM)pszBuf) != CB_ERR) {
				char latedata[256];
				int len = strlen(IniFilepath);
				// IniFileを指定していない場合は、変数の長さが0になるので、デフォルトのIniFileを指定する
				if (len == 0){
					char Path[MAX_PATH + 1];
					char settingpath[MAX_PATH + 1];
					if (GetModuleFileName(NULL, Path, MAX_PATH) != 0) {
						char drive[MAX_PATH + 1], dir[MAX_PATH + 1], fname[MAX_PATH + 1], ext[MAX_PATH + 1];
						// パス名を分解
						_splitpath(Path, drive, dir, fname, ext);
						_stprintf_s(settingpath, MAX_PATH + 1, INIFILEDEFNAME, drive, dir);
						//MessageBox(hwnd, settingpath, _T("結果の出力先の確認"), MB_OK);
					}
					// iniファイルの読み込み
					GetPrivateProfileString(pszBuf, _T("data"), _T("Error"), latedata, sizeof(latedata), settingpath);
				}
				else {
					GetPrivateProfileString(pszBuf, _T("data"), _T("Error"), latedata, sizeof(latedata), IniFilepath);
				}

				// コンボボックスの中身を消去
				while (SendMessage(GetDlgItem(hwnd, (int)ID_LATESETTING), CB_GETCOUNT, 0, 0)){
					SendMessage(GetDlgItem(hwnd, (int)ID_LATESETTING), CB_DELETESTRING, 0, 0);
				}

				stringstream ss, tt;
				ss << latedata;
				string s, t;

				int i = 0;
				while (getline(ss, s, ',')) {
					char* cstr = new char[s.size() + 1];
					char_traits<char>::copy(cstr, s.c_str(), s.size() + 1);
					// コンボボックスに文字列を挿入
					SendMessage(GetDlgItem(hwnd, ID_LATESETTING), CB_INSERTSTRING, i, (LPARAM)cstr);
					i++;
				}
				// コンボボックスに先頭の要素をセット
				SendMessage(GetDlgItem(hwnd, ID_LATESETTING), CB_SETCURSEL, 0, 0);
			}
			
			delete[] pszBuf;
		}
	
	return true;
}
// コンボボックス1で選択されたキー名に対応した遅延時間をコンボボックスに詰める。
bool SendLate2Combo_2(HWND hwnd, WPARAM wParam, char* IniFilepath) {


	//コンボボックスで現在選択されている項目のインデックスを取得
	int intCurrentIndex = SendMessage(GetDlgItem(hwnd, (int)ID_LATEINI_2), CB_GETCURSEL, 0, 0);

	// 現在選択されている項目のの文字列の長さを取得
	int intTxtLen = SendMessage(GetDlgItem(hwnd, (int)ID_LATEINI_2), CB_GETLBTEXTLEN, intCurrentIndex, 0);

	if (intTxtLen != CB_ERR) {
		char* pszBuf = new char[intTxtLen + 1];
		if (SendMessage(GetDlgItem(hwnd, ID_LATEINI_2), CB_GETLBTEXT, intCurrentIndex, (LPARAM)pszBuf) != CB_ERR) {
			char latedata[256];
			if (IniFilepath[0] == _T('\0')) {
				char Path[MAX_PATH + 1];
				char settingpath[MAX_PATH + 1];
				if (GetModuleFileName(NULL, Path, MAX_PATH) != 0) {
					char drive[MAX_PATH + 1], dir[MAX_PATH + 1], fname[MAX_PATH + 1], ext[MAX_PATH + 1];
					// パス名を分解
					_splitpath(Path, drive, dir, fname, ext);
					_stprintf_s(settingpath, MAX_PATH + 1, INIFILEDEFNAME, drive, dir);
				}
				// iniファイルの読み込み
				GetPrivateProfileString(pszBuf, _T("data"), _T("Error"), latedata, sizeof(latedata), settingpath);
			}
			else {
				GetPrivateProfileString(pszBuf, _T("data"), _T("Error"), latedata, sizeof(latedata), IniFilepath);
			}
			// コンボボックスの中身を消去
			while (SendMessage(GetDlgItem(hwnd, (int)ID_LATESETTING_2), CB_GETCOUNT, 0, 0)) {
				SendMessage(GetDlgItem(hwnd, (int)ID_LATESETTING_2), CB_DELETESTRING, 0, 0);
			}

			stringstream ss, tt;
			ss << latedata;
			string s, t;

			int i = 0;
			while (getline(ss, s, ',')) {
				char* cstr = new char[s.size() + 1];
				char_traits<char>::copy(cstr, s.c_str(), s.size() + 1);
				// コンボボックスに文字列を挿入
				SendMessage(GetDlgItem(hwnd, ID_LATESETTING_2), CB_INSERTSTRING, i, (LPARAM)cstr);
				i++;
			}
			// コンボボックスに先頭の要素をセット
			SendMessage(GetDlgItem(hwnd, ID_LATESETTING_2), CB_SETCURSEL, 0, 0);
		}

		delete[] pszBuf;
	}

	return true;
}

// コンボボックスで現在選択されている項目の文字列を取得する関数
int GetNowComboStr(HWND hWnd, int comboID) {

	int LocalNum;
	//コンボボックスで現在選択されている項目のインデックスを取得
	int intCurrentIndex = SendMessage(GetDlgItem(hWnd, (int)comboID), CB_GETCURSEL, 0, 0);

	// コンボボックスの現在選択されている項目の文字列の長さを取得
	int intTxtLen = SendMessage(GetDlgItem(hWnd, (int)comboID), CB_GETLBTEXTLEN, intCurrentIndex, 0);

	if (intTxtLen != CB_ERR) {
		char* pszBuf = new char[intTxtLen + 1];
		// コンボボックスの一覧から選択した項目の文字列を取得
		if (SendMessage(GetDlgItem(hWnd, (int)comboID), CB_GETLBTEXT, intCurrentIndex, (LPARAM)pszBuf) != CB_ERR) {
			// char*型をint型に変換
			 LocalNum = atoi(pszBuf);
		}

		delete[] pszBuf;
	}
	return LocalNum;
}


// コンボボックスの選択項目から遅延時間を算出
bool CalcLateNumberOfloops(HWND hwnd, int* NumberOfloops, int  Num, int lenbuffer, int rate, long inlatency, long outlatency){
	/* 引数: ウィンドウハンドル, コンボボックスから得る文字列, バッファ長, サンプリング周波数, 
	*          インプットレイテンシ, アウトプットレイテンシ
	*/
	////////////////////////
	// 変数をms単位に変換
	//////////////////////////
	TCHAR NumberOfloopsInfo[LENSTR];
	// 入力レイテンシー(バッファ2個分の時間を減算)
	double inlatency_ms = (double)inlatency * 1000.0 / (double)rate - (double)lenbuffer * 2.0 * 1000.0 / (double)rate;
	// 出力レイテンシー
	double outlatency_ms = (double)outlatency * 1000.0 / (double)rate;
	// マイコンでの遅延
	double kairo = (double)KAIROLATE;

	// NumberOfloopsの計算
	*NumberOfloops = ((double)Num - (inlatency_ms + outlatency_ms + kairo)) * (double)rate / ((double)lenbuffer * 1000.0);

	// NumberOfloops == 0のとき、1に更新
	if (*NumberOfloops < 1) *NumberOfloops = 1;


	////メッセージボックスへの出力
	//_stprintf_s(NumberOfloopsInfo, LENSTR,
	//	_T("入力レイテンシ: %f [ms]\n"
	//	"出力レイテンシ: %f [ms]\n"
	//	"入出力レイテンシ: %f [ms]\n"
	//	"遅延させたい時間: %d [ms]\n"
	//	"回路による遅延: %f [ms]\n"
	//	"バッファ長: %d [points]\n"
	//	"Fs: %d [Hz]\n"
	//	"NumberOfloops = %d\n"
	//	"NumberOfloops(double) = %f"),
	//	inlatency_ms, outlatency_ms, inlatency_ms + outlatency_ms, Num, kairo, lenbuffer, rate, *NumberOfloops,
	//	((double)Num - (inlatency_ms + outlatency_ms + kairo)) * (double)rate / ((double)lenbuffer * 1000.0));

	//MessageBox(hwnd, NumberOfloopsInfo, _T("計算結果"), MB_OK);

	return true;

}

// エディットボックスを自動的に下までスクロールする
bool ScrollToBottom(HWND hEditBox) {
	// エディットボックス内の文字列の長さを取得
	int len = GetWindowTextLength(hEditBox);
	// 何も選択せずにエディットボックス内のテキストの最後にキャレットを移動
	SendMessage(hEditBox, EM_SETSEL, (WPARAM)len, (LPARAM)len);
	// エディットボックスをカーソルまでスクロールする
	SendMessage(hEditBox, EM_SCROLLCARET, 0, 0);

	return true;
}

// 被験者情報をベクタに保存
bool GetEditBoxTextFunc(HWND hwnd) {
	char* lpstringOld = new char[10];
	char* lpstringName = new char[50];

	GetWindowText(GetDlgItem(hwnd, ID_EDIT_OLD), lpstringOld, 10);
	GetWindowText(GetDlgItem(hwnd, ID_EDIT_NAME), lpstringName, 50);

	string Old = lpstringOld;
	string Name = lpstringName;

	MarginTime.insert(MarginTime.begin(), Old);
	MarginTime.insert(MarginTime.begin(), Name);

	delete[] lpstringOld;
	delete[] lpstringName;

	return true;
}

bool OnPaint(HWND hWnd, HDC hdc) {
	HPEN hBluePen = CreatePen(PS_SOLID, 5, (RGB(0, 150, 215)));
	RECT rButtonSendCSV = {rGroupIni.left + 294 , 374, rGroupIni.left + 294  + rGroupIni2.right - rGroupIni2.left + 52, 373 + 52
	};
	HPEN hOldPen = (HPEN)SelectObject(hdc, hBluePen);
	Rectangle(hdc, rButtonSendCSV.left, rButtonSendCSV.top, rButtonSendCSV.right, rButtonSendCSV.bottom);
	SelectObject(hdc, hOldPen);

	DeleteObject(hBluePen);
	return true;
}
bool OnCommand(HWND hwnd, WPARAM wparam, LPARAM lparam) {

	switch (LOWORD(wparam)) {
	case ID_ASIOINFORMATION:
		//ASIO情報の表示
		MessageBox(NULL, strASIOInfo, _T("ASIO"), MB_OK);
		break;

	case ID_WAVEFILEINFORMATION:
		//Waveファイル情報の表示
		MessageBox(NULL, strWaveFileInfo, _T("Wave File"), MB_OK);
		break;

	case ID_EXIT:
		//プログラムの終了
		if (IDYES == MessageBox(hwnd, _T("終了しますか?"), _T("終了確認"), MB_YESNO)) {
			DestroyWindow(hwnd);
		}
		break;
		
		// 遅延時間の設定
		// 配列の削除
	case ID_MENU_RESTART:
		if (MessageBox(hwnd, _T("結果をファイルに書き込みましたか?"), _T("確認"), MB_YESNO) == IDYES) {
			if (ClearMarginTime(hwnd, MarginTime)) {
				MessageBox(hwnd, _T("結果を削除しました。"), _T("削除完了"), MB_OK);
			}
			// コンボボックスの有効化
			if (!LABNormal)
			{
				EnableWindow(GetDlgItem(hwnd, ID_EDIT_LATEDATA_TIMING), TRUE);
				EnableWindow(GetDlgItem(hwnd, ID_LATEINI_2), TRUE);
				EnableWindow(GetDlgItem(hwnd, ID_LATESETTING_2), TRUE);
			}
			else {
				EnableWindow(GetDlgItem(hwnd, ID_LATESETTING), TRUE);        
				EnableWindow(GetDlgItem(hParentWindow, ID_LATEINI), TRUE);
			}
			if (Finished == true) { Finished = false; }
		}
		break;

	case ID_MENU_INI_FILE:
		// ファイル選択のためのダイアログボックスを表示
	{
		bool ReadIniFileFlag = SelectIniFile(hwnd, FileNameINI);
		if (ReadIniFileFlag) {
			// Iniファイルの更新
			ReadIniFile(hwnd, lparam, FileNameINI);
			// ループ回数の設定
			LateIniFunc(hwnd, wparam);
			// ウィンドウに表示されるファイル名の更新
			RelativepathFromAbsolutepath(FileNameINI, hwnd, ID_GROUP_LATEDATA, false);
			RelativepathFromAbsolutepath(FileNameINI, hwnd, ID_GROUP_LATEDATA_2, true);

		}
		break;
	}
	case ID_DELETE_ARRAY:
		/*NumberOfloops = 1;
		SetWindowText(hStaticNumberOfloops2, _T("1"));*/
		if (ClearMarginTime(hwnd, MarginTime)) {
			MessageBox(hwnd, _T("結果を削除しました。"), _T("削除完了"), MB_OK);
		}
		break;
		///////////////////
		// コンボボックス
		///////////////////
	case ID_LATEINI:
		// 基本情報
		if (HIWORD(wparam) == CBN_SELCHANGE) {
			SendLate2Combo(hwnd, wparam, FileNameINI);
			LateIniFunc(hwnd, wparam);
		}
		break;
		
	case ID_LATESETTING:
		// 詳細情報
		if (HIWORD(wparam) == CBN_SELCHANGE) {
			LateIniFunc(hwnd, wparam);
		}
		break;		break;
		////////////////////
		// コンボボックス(変則var)
		////////////////////

	case ID_LATEINI_2:
		// 基本情報(変則var)
		if (HIWORD(wparam) == CBN_SELCHANGE) {
			SendLate2Combo_2(hwnd, wparam, FileNameINI);
			LateIniFunc_2(hwnd, wparam);
		}
		break;
	case ID_LATESETTING_2:
		// 詳細情報(変則var)
		if (HIWORD(wparam) == CBN_SELCHANGE) {
			LateIniFunc_2(hwnd, wparam);
		}
		break;

	// 取得回数の設定
	case ID_STATIC_TRIALS_NUM:
		if (HIWORD(wparam) == CBN_SELCHANGE) {
			int itemIndex = SendMessage(GetDlgItem(hwnd, ID_STATIC_TRIALS_NUM), CB_GETCURSEL, 0, 0);
			if (itemIndex != CB_ERR) {
				char itemData[10] = {0};
				SendMessage(GetDlgItem(hwnd, ID_STATIC_TRIALS_NUM), CB_GETLBTEXT, itemIndex, (LPARAM)itemData);
				int Temp = _ttol(itemData);
				TempNumberOfTrials = Temp;
			}
		}
		break;
		// 実験方法の選択
	case ID_STATIC_NORMALORIRREG:
		if (HIWORD(wparam) == CBN_SELCHANGE) {
			int itemIndex = SendMessage(GetDlgItem(hwnd, ID_STATIC_NORMALORIRREG), CB_GETCURSEL, 0, 0);
			if (itemIndex != CB_ERR) {
				if (itemIndex) {
					/*if (DialogBox(GetModuleHandle(NULL), MAKEINTRESOURCE(IDD_MYDIALOG), hwnd, DialogProc) == -1) {
						MessageBox(hwnd, _T("ダイアログボックスの作成に失敗しました。"), _T("エラー"), MB_OK);
					}*/
					EnableWindow(GetDlgItem(hwnd, ID_EDIT_LATEDATA_TIMING), TRUE);
					EnableWindow(GetDlgItem(hwnd, ID_LATEINI_2), TRUE);
					EnableWindow(GetDlgItem(hwnd, ID_LATESETTING_2), TRUE);
					EnableWindow(GetDlgItem(hwnd, ID_LATESETTING), FALSE);
					EnableWindow(GetDlgItem(hParentWindow, ID_LATEINI), FALSE);
					// 遅延のタイミング
					char* lpstringDelayTiming = new char[50];
					GetWindowText(GetDlgItem(hwnd, ID_EDIT_LATEDATA_TIMING), lpstringDelayTiming, 50);
					DelayTiming = atoi(lpstringDelayTiming);
					//コンボボックスで現在選択されている項目のインデックスを取得
					int LocalDelayTIme = GetNowComboStr(hwnd, ID_LATESETTING_2);
					DelayTime_ms = LocalDelayTIme;

					LABNormal = false;
					delete[] lpstringDelayTiming;
				}
				else {
					EnableWindow(GetDlgItem(hwnd, ID_EDIT_LATEDATA_TIMING), FALSE);
					EnableWindow(GetDlgItem(hwnd, ID_LATEINI_2), FALSE);
					EnableWindow(GetDlgItem(hwnd, ID_LATESETTING_2), FALSE);
					EnableWindow(GetDlgItem(hwnd, ID_LATESETTING), TRUE);
					EnableWindow(GetDlgItem(hParentWindow, ID_LATEINI), TRUE);

					DelayTiming = 1;
					LABNormal = true;
				}
			}
		}
		break;
		// BPMの選択
	case ID_STATIC_BPMNUM:
		if (HIWORD(wparam) == CBN_SELCHANGE) {
			int itemIndex = SendMessage(GetDlgItem(hwnd, ID_STATIC_BPMNUM), CB_GETCURSEL, 0, 0);
			if (itemIndex != CB_ERR) {
				/*BPMの変更に伴う処理をここに書く*/
			}
		}
		break;
	////////////////////////////////
	// 「メニュー」→「実験方法」
	// /////////////////////////////
	case ID_MENU_NORMAL:
		SetWindowText(GetDlgItem(hParentWindow, ID_STATIC_NORMALORIRREG), _T("通常"));
		LABNormal = true;
		break;
	case ID_MENU_IRREG:
		SetWindowText(GetDlgItem(hParentWindow, ID_STATIC_NORMALORIRREG), _T("変則"));
		LABNormal = false;
		break;
	
		///////////////
	// ファイルの指定
		////////////////
	case ID_MENU_CSV_OUTPUT:
	{
		// ファイル選択用ダイアログの表示
		bool MadeFile = SelectFile(hwnd, FileNameCSV);
		if (MadeFile)
		{
			// パス名をstring型に変換
			CSVFILENAME = FileNameCSV;
			// 出力先ファイルをウィンドウ上に表示
			RelativepathFromAbsolutepath(FileNameCSV, hwnd, ID_EDIT_FILEPATH, false);
		}
	}
		break;
		// プッシュボタン
	case ID_BUTTON_CSV:
		if (HIWORD(wparam) == BN_CLICKED) {
			Finished = false; // ファイルに書き込んだらtrueにする
			// 年齢と名前が書かれているかチェック
			short length_1 = GetWindowTextLength(GetDlgItem(hwnd, ID_EDIT_OLD));
			short length_2 = GetWindowTextLength(GetDlgItem(hwnd, ID_EDIT_NAME));
			if (length_1 * length_2 == 0) {
				// Edit Boxが空
				MessageBox(hwnd, _T("名前と年齢のどちらか、もしくはその両方が未入力です。"), _T("警告"), MB_OK | MB_ICONWARNING);
				break;
			}
			// 既に開かれているかをチェック
			if (!CheckCanWriteFile(hwnd, FileNameCSV)) {
				// ファイルを開くのに失敗した場合、書き込みが不可能である可能性が高い
				MessageBox(hwnd, _T("ファイルに書き込めません。ファイルが選択されていないか既に開かれている可能性があります。"), _T("警告"), MB_OK | MB_ICONWARNING);
				break;
			}
			// 遅延時間のグループ名とBPMを出力
			 // リサイズ。書き込み可能となるようにメモリを確保
			string delay_group,bpm;
			delay_group.resize(10);  
			bpm.resize(10);
			// null文字を除いた書き込んだ文字数を返す
			int NumWritten=0;
			if (!LABNormal) {
				NumWritten = GetWindowText(GetDlgItem(hwnd, ID_LATEINI_2), &delay_group[0], sizeof(delay_group));
			}else{
				NumWritten = GetWindowText(GetDlgItem(hwnd, ID_LATEINI), &delay_group[0], sizeof(delay_group));
			}
			int NumBpm = GetWindowText(GetDlgItem(hwnd, ID_STATIC_BPMNUM), &bpm[0], sizeof(bpm));
			// null文字を除いた要素数にresizeする
			delay_group.resize(NumWritten);       
			bpm.resize(NumBpm);
			
			MarginTime.insert(MarginTime.begin(), bpm);                                // BPMをベクタの先頭に代入
			GetEditBoxTextFunc(hParentWindow);                                            // 被験者情報をベクタの先頭に代入
			MarginTime.insert(MarginTime.begin(), delay_group);                     // 遅延時間のグループ名をベクタの先頭に代入
			MarginTime.insert(MarginTime.begin(), to_string(Num));                 // 遅延時間をベクタの先頭に代入
			MarginTime.insert(MarginTime.begin(), TimeStampButtonClicked);   // 押下時刻をベクタの先頭に代入
			if (!LABNormal) {
				string tempunique = _T("irregular");
				string stringDelayTiming = to_string(DelayTiming);
				string stringDelayTime_ms = to_string(DelayTime_ms);

				// 実験条件が変則の場合、ベクタの先頭に文字列を代入
				MarginTime.insert(MarginTime.begin(), stringDelayTiming);        // 遅延のタイミング
				MarginTime.insert(MarginTime.begin(), stringDelayTime_ms);    // 遅延時間
				MarginTime.insert(MarginTime.begin(), tempunique);                // irregular
				EnableWindow(GetDlgItem(hwnd, ID_LATESETTING_2), TRUE);
				EnableWindow(GetDlgItem(hParentWindow, ID_LATEINI_2), TRUE);
				EnableWindow(GetDlgItem(hParentWindow, ID_EDIT_LATEDATA_TIMING), TRUE);
			}
			// コンボボックスの更新
			// コンボボックスの表示を次の選択肢に更新
			//if (LABNormal) {
			//	int NumberOfCursel = SendMessage(GetDlgItem(hParentWindow, ID_LATESETTING), CB_GETCOUNT, 0, 0); // 選択肢の数
			//	int NowCursel = SendMessage(GetDlgItem(hParentWindow, ID_LATESETTING), CB_GETCURSEL, 0, 0);        // 現在の選択項目
			//	if (NowCursel == NumberOfCursel - 1)
			//		SendMessage(GetDlgItem(hParentWindow, ID_LATESETTING), CB_SETCURSEL, 0, 0); // 先頭項目に戻す
			//	else SendMessage(GetDlgItem(hParentWindow, ID_LATESETTING), CB_SETCURSEL, NowCursel + 1, 0); // 次の項目に更新
			//}
			//else
			//{
			//	int NumberOfCursel = SendMessage(GetDlgItem(hParentWindow, ID_LATESETTING_2), CB_GETCOUNT, 0, 0); // 選択肢の数
			//	int NowCursel = SendMessage(GetDlgItem(hParentWindow, ID_LATESETTING_2), CB_GETCURSEL, 0, 0);        // 現在の選択項目
			//	if (NowCursel == NumberOfCursel - 1)
			//		SendMessage(GetDlgItem(hParentWindow, ID_LATESETTING_2), CB_SETCURSEL, 0, 0); // 先頭項目に戻す
			//	else SendMessage(GetDlgItem(hParentWindow, ID_LATESETTING_2), CB_SETCURSEL, NowCursel + 1, 0); // 次の項目に更新
			//}
			WriteToCSV(hwnd, MarginTime, CSVFILENAME);
			EnableWindow(GetDlgItem(hwnd, ID_LATESETTING), TRUE);        // コンボボックスの有効化
			EnableWindow(GetDlgItem(hParentWindow, ID_LATEINI), TRUE);

		}
		break;
	}

	return true;
}

// 不偏標準偏差を計算するための関数
double calculate_unbiased_stddev(const vector<double>& numbers) {

	// データの平均値を算出
	double ave = 0.0;
	for (const auto& num : numbers) {
		ave += num;
	}
	ave /= numbers.size();

	double variance = 0.0;
	for (const auto& num : numbers) {
		variance += (num - ave) * (num - ave);
	}
	// n-1で割り不偏分散を計算
	variance /= (numbers.size() - 1);
	
	// 最後に不偏分散の平方根をとる
	return sqrt(variance);
}

// string vectorから不偏標準偏差を計算するための関数
bool SetStddev2Window(const vector<string> mTime) {

	vector<double> numbers;

	// 文字列を数値に変換、vectorに格納
	for (const auto& str : mTime) {
		if (str == "false" ) {
			// "false"だった場合はスキップ
			continue;
		}
		numbers.push_back(stod(str));
	}
	// 不偏標準偏差を取得
	double std_dev = calculate_unbiased_stddev(numbers);
	// 小数点以下1桁までに丸める(中間の値(2.5や3.5など)は、最も近い偶数整数に四捨五入される)
	std_dev = round(std_dev * 10.0) / 10.0;

	// string型の文字列に変換
	stringstream ss;
	ss << std_dev;
	string std_dev_str = ss.str();

	// ウィンドウに結果を表示
	std_dev_str = _T("Std[ms]: ") + std_dev_str;
	SetWindowText(GetDlgItem(hParentWindow, ID_STATIC_STDDEV), std_dev_str.c_str());

	return true;
}
\end{lstlisting}
	