\section{主観評価におけるアプリケーション開発}
先行研究\cite{cf:kayama}で著者らは,

本節では,2.1節で述べた主観評価実験で利用することを想定したアプリケーションの概要と機能について説明する.
これまでの主観評価実験では、遅延聴覚フィードバック下での違和感を被験者に実験者が用意した用紙に入力させ,その後PC上にデータを移行して結果を保存していた.
このデータ移行には入力ミスのリスクがあり,注意深い作業が必要で研究にとって非効率であった.
このアプリケーションを作成することで,被験者がアプリケーション上に直接評価結果を入力し,自動で外部ファイルに結果を出力できるようになる.これにより,実験者の負担が軽減され,効率的な実験および評価結果の分析が可能となり,短期間で多くの実験を実施することが期待できる.
また,本節で述べる関数は,文献\cite{Win32API-reference}に基づいて利用する.
