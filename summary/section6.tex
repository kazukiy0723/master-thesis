\section{結論}
\subsection{まとめ}
本研究では、発話時の遅延聴覚フィードバックの影響を調査するためのアプリケーションを開発し、
被験者が直接評価結果を入力し外部ファイルに出力する機能を備えた。
また、ボタン押し課題を用いて、
聴覚フィードバックの遅延が身体運動に与える影響を客観的に評価するための改良されたシステムを利用した。
遅延聴覚フィードバックの影響を観察するため、特定の条件下でボタン押し課題を行い、
その結果を分析した。若年者と高齢者を対象にした調査から、
遅延時間に対する感受性に年齢による違いがあることが明らかになった.
若年者は遅延時間に対して敏感である一方で、高齢者は遅延時間に対して一定の許容度を持っている可能性が示された.
\subsection{今後の課題}
今後は,高齢者と若年者の運動能力の差異を考慮し、遅延聴覚フィードバックの影響を公平に評価するために、運動能力に応じた課題の検討が必要である。
また,遅延聴覚フィードバックが発話に及ぼす影響の客観評価方法の検討および本研究で得られたデータとの比較も必要である。
これらは、補聴器の設計に役立つ知見を提供することが期待される.