\section{結論}
\subsection{まとめ}
本研究では,遅延聴覚フィードバックに起因する違和感を客観的に評価するために,文献\cite{cf:shigematu}で開発されたシステムを改良し,
また,ボタン押し課題を用いた客観的な評価方法を開発した.
そして,これらを用いて遅延聴覚フィードバックの身体運動への影響を調査した.
遅延聴覚フィードバックの影響をより明確に観察するため,特定の条件下でボタン押し課題を行い,
その結果を分析した.
若年者と高齢者を対象にした調査から,
聴覚フィードバックの遅延時間に対する感受性において年齢による違いがあることが明らかになった.
若年者は遅延時間に対して敏感である一方で,高齢者は遅延時間に対して一定の許容度を持っている可能性が示唆された.
加えて,高齢者が若年者に比べてすべての遅延時間で高い評価値を示したという結果は,
高齢者と若年者間の潜在的な運動能力の差異がこれらの反応に影響を及ぼしている可能性を示唆している.
\subsection{今後の展望}
今後は,高齢者と若年者の運動能力の差異を考慮し,遅延聴覚フィードバックの影響を公平に評価するために,
運動能力に応じた課題の検討が必要であると考えられる.
また,遅延聴覚フィードバックが発話に与える影響の客観的な評価方法の検討および本研究で得られたデータとの比較も必要である.
% これらは,補聴器の設計に役立つ知見を提供することが期待される.
これらのアプローチを通じて,遅延聴覚フィードバックの影響をより深く理解し,それが補聴器の設計に繋がることが期待される.