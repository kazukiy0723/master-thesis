\section{序論}
\subsection{背景}
% 本節では,ディジタル補聴器の進化と課題について述べる.
ディジタル補聴器は,ディジタル信号処理技術を駆使し,従来のアナログ補聴器に比べ高度な機能を実現しているものの,
利用者の満足度に関しては依然として問題が指摘されている\cite{cf:Manzokudo}.
性能の向上には,精緻なディジタル信号処理および周波数帯域の細分化が求められるが,これにはフレーム長の増加という課題が伴う.
フレーム長が長くなることで,補聴器の音声の入出力間の遅延時間が増大し,遅延という新たな問題が生じる.
人間は能動的な活動を行う際,活動とそれに伴う感覚フィードバックを対応付けることで行動の調整を行っている.
この中で,聴覚に関するフィードバックを聴覚フィードバックと呼び,
特に自己の発声などが遅れて聴覚にフィードバックされることを遅延聴覚フィードバックと呼ぶ\cite{cf:DAF}.
一般に,聴覚フィードバックの遅延が10[ms]を超えると発話や身体運動に影響が出るとされ\cite{cf:DelayTime-ninnchi},
ディジタル補聴器における遅延もこれに該当するため,遅延を短縮しつつ高度な処理を実現することは困難である.
しかし,高齢者は聴覚フィードバックの遅延時間が10[ms]を超えても違和感を覚えにくいことが示唆されている\cite{shigematu-toukyoushibu}.
この知見を活用して遅延時間を増大させることで,高齢者向け補聴器におけるより高度なディジタル信号処理の実装が期待される.
\subsection{目的}
本研究では,若年者と高齢者の聴覚フィードバックの遅延時間の許容範囲の差について検討する.
遅延聴覚フィードバックによる違和感を客観的に評価する目的で,遅延聴覚フィードバックが身体運動に与える影響を調査する.
ここでは,身体運動への影響を幅広い年代で比較することを想定して,簡易なボタン押し課題を採用し,遅延の影響を分析する.
本研究で行うボタン押し課題は,一定の時間間隔ごとにコントローラのボタンを押すという動作を一定の回数行う課題である.
先行研究\cite{cf:kayama}では,遅延聴覚フィードバックが発話に与える影響についての調査がされたが,
この調査は主観的な評価に基づくもので,個人差が顕著であるという問題があった.
そこで,本研究では,遅延聴覚フィードバックによる影響を客観的に評価するため,
先行研究\cite{cf:shigematu}で使用された調査のシステムに改良を加え,改良したボタン押し課題のシステムを用いて,遅延聴覚フィードバックの身体運動への影響を調査する.
遅延聴覚フィードバックによる影響の大きさをより明確に観察するため,本研究のボタン押し課題における最適な条件を検討し,若年者と高齢者を対象に身体運動への影響の調査を行う.
本研究は,聴覚フィードバックの遅延が身体運動に与える影響と年齢差の関係を明らかにし,
高齢者向け補聴器の設計において重要な示唆を提供することが期待される.