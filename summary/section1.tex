\section{序論}
\subsection{背景}
% 本節では,ディジタル補聴器の進化と課題について述べる.
ディジタル補聴器は,ディジタル信号処理技術を駆使し,従来のアナログ補聴器に比べ高度な機能を実現しているものの,
利用者の満足度に関しては依然として問題が指摘されている\cite{cf:Manzokudo}.
性能の向上には,精緻なディジタル信号処理および周波数帯域の細分化が求められるが,これにより音声信号の長さが増加し,
遅延時間という新たな問題が生じている.人間は能動的な活動を行う際,活動とそれに伴う感覚フィードバックを対応付けることで
行動の調整を行っている.この中で,聴覚に関するフィードバックを聴覚フィードバックと呼び,
特に遅れた聴覚フィードバックを遅延聴覚フィードバックと呼ぶ\cite{cf:DAF}.
一般に聴覚フィードバックの遅延時間が10[ms]を超えると,発話や身体運動に影響を与えることが知られている\cite{cf:DelayTime-ninnchi}.
特に,ディジタル補聴器における遅延時間もこの遅延時間に該当し,
この遅延時間を短縮しつつ高度な処理を実現することが困難である.
しかし,高齢者は遅延時間が10[ms]を超えても違和感を覚えにくいことが示唆されている\cite{shigematu-toukyoushibu}.
この知見を利用して遅延時間を増大させることで,高齢者用補聴器においてより高度なディジタル信号処理を実装することが期待される.
\subsection{目的}
本研究では,若年者と高齢者の聴覚フィードバックの遅延時間の許容量の差を調査し,
聴覚フィードバックによる違和感を客観的に評価するため,聴覚フィードバックの
遅延が身体運動に与える影響を検討する.
身体運動への影響を幅広い年代で比較することを想定して,簡易なボタン押し課題を採用する.
この課題では,メトロノームの合図音に合わせてボタンを押す動作を行い,遅延の影響を分析する.
先行研究\cite{cf:kayama}では,遅延聴覚フィードバックが発話に与える影響についての調査がされたが,
この調査は主観的な評価に基づくもので,個人差が顕著であるという問題があった.
そこで,本研究では,遅延聴覚フィードバックによる影響を客観的に評価するため,
先行研究\cite{cf:shigematu}で著者らによって行われた調査のシステムについて改良を行い,
遅延聴覚フィードバックの身体運動への影響を調査することで,客観的な評価方法を開発することを目指す.
遅延聴覚フィードバックによる影響の大きさを探るため,本研究のボタン押し課題の最適な条件を検討し,若年者と高齢者を対象に影響の調査を行う.
本研究は,聴覚フィードバックの遅延が身体運動に与える影響と年齢差の関係を明らかにし,
高齢者向け補聴器の設計において重要な示唆を提供することが期待される.