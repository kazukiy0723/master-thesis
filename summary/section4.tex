\section{遅延聴覚フィードバックが身体運動に与える影響の評価方法}
遅延聴覚フィードバックが身体運動に与える影響の評価は,被験者が行うボタン押下の時間間隔の分散と
遅延が4の倍数に到達したときのみ発生する状況を考慮して,4の倍数に到達する直前のボタンの押下時間間隔と
直後のボタン押下間隔のデータの差の二乗平均(Mean Squared Error, MSE)および誤差の中央値(Median Squared Error, MedSE)を用いて行う.
この評価方法は,遅延聴覚フィードバックが身体運動に影響を与えている場合,遅延が発生する直前のボタン押下間隔と
直後のボタン押下間隔の差が大きくなることを想定している.
% 以下にこれらの評価指標の定義を示す.
% \begin{equation}
%   s^2_a = \frac{1}{l-1} \sum_{i=k}^{k+l-1} (x_i - \bar{x}_{kl})^2
% \end{equation}
% \begin{equation}
%   s^2_b = \frac{1}{l} \sum_{i=k}^{k+l-1} (x_i - M_{kl})^2
% \end{equation}
% \begin{equation}
%   s^2_c = \frac{1}{l} \sum_{i=k}^{k+l-1} (x_i - T)^2
% \end{equation}
% \begin{equation}
%   f(n) = \left\lfloor \frac{n-t-1}{t} \right\rfloor + 1
% \end{equation}
% \begin{equation}
%   s(n) = \left\lfloor \frac{n-t-2}{t-1} \right\rfloor
% \end{equation}