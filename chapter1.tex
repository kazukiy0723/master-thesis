\chapter{序論}
\section{背景}
本節では,現代における補聴器技術の主流であるディジタル補聴器に着目し,その進化と課題について述べる.
ディジタル補聴器はディジタル信号処理技術を活用し,従来のアナログ補聴器に比べて高度な機能を実現していることが指摘されている\cite{EaringAids_20}\cite{EaringAids-2}.
具体的には,音声の増幅率を音圧レベルに応じて調整するノンリニア増幅機能や,雑音を低減しつつ目的の音声を強調するノイズリダクション機能などが挙げられる.
しかしながら,補聴器の利用者からは,これらの機能にも関わらず十分な満足度が得られていないという問題が報告されている\cite{Manzokudo}.
不満の一因として,ノンリニア増幅が不要な音まで増幅することに起因するという意見が存在する.
このことは,ディジタル補聴器のさらなる性能改善への要求を示唆している.ディジタル補聴器の性能を向上させるためには,精緻なディジタル信号処理が必要であり,これには音声信号の周波数帯域を細分化することが求められる.
しかしながら,周波数帯域の細分化は,処理に使用するフレーム長の大きさを増加させるという課題を伴う.
ディジタル補聴器では,音声信号を数ミリ秒単位のフレームに分割して処理を行うため,入力から出力までにフレーム長に相当する時間の遅延が発生する.
さらに,アナログ信号をディジタル信号に変換し,その後再びアナログに戻すAD/DA変換プロセスによっても遅延が生じる.
このAD/DA変換とフレーム長に起因する遅延は,音声の入出力間で最低数ミリ秒のタイムラグを発生させる.
したがって,ディジタル信号処理に用いるフレーム長が長くなると,補聴器における音声の入出力信号間の遅延時間も増大することになる.
人間は能動的な活動を行う際,活動とそれに伴う感覚フィードバックを対応付けることで行動の調整を行っている.
この中で,聴覚に関するフィードバックを聴覚フィードバックと呼んでいる\cite{DAF}.聴覚フィードバックは,発話や身体運動において重要な役割を果たしている.
例として,発話時に自身の声を聴くことにより,音声の高さや強さを調整する行為が挙げられる.
遅延聴覚フィードバックは,発話者が自身の声を聴く際のタイミングが遅延することによって生じ,ディジタル補聴器における入出力信号間の遅延がこれに該当する.
一般に,この遅延時間が10[ms]を超えると発話に違和感を覚えることが知られている\cite{DelayTime-ninnchi}.そのため,ディジタル補聴器はこの遅延時間を超えないよう設計されている\cite{EaringAids_20}.
しかし,この設計制約がディジタル補聴器の性能向上における課題となっている.
ディジタル信号処理の精度を高めるためにはフレーム長を長くする必要があり,これは必然的に入出力間の遅延時間を増加させる.
そのため,ディジタル信号処理における高度な処理と遅延時間の短縮という二つの要求を両立させることが困難である.
一方で,高齢者においては,遅延時間が10[ms]を超えても発話に違和感を覚えにくいことが示唆されている\cite{shigematu-toukyoushibu}.
この知見を活用すれば,ディジタル補聴器の入出力信号間の遅延時間を増大させ,それに伴いフレーム長を長くすることが可能となる.
これにより,周波数帯域の細分化を進め,より高度なディジタル信号処理を実装することが期待される.
このアプローチは,特に高齢者における補聴器の性能向上に寄与する可能性がある.
\section{目的}
本研究では,若年者と高齢者における聴覚フィードバックの遅延時間の許容範囲の差について検討する.
文献\cite{kayama}では,遅延聴覚フィードバックが発話に与える影響の調査が行われたが,この調査は主観的な評価によるものであるため,個人差の影響を軽減する客観的な評価方法の必要性が指摘されている.
そこで,文献\cite{shigematu}において,著者らが行った調査のシステムについて改良を行い,遅延聴覚フィードバックの身体運動への影響を調査することで,客観的な評価方法を開発することを目指す.
ここでは,身体運動への影響を幅広い年代の被験者間で比較することを想定して,高齢者でも簡単に実験を行うことのできるボタン押し課題を採用する.
本研究で行うボタン押し課題は,一定の時間間隔ごとにコントローラのボタンを押すという動作を一定の回数行う課題である.
被験者がボタンを押下する一定の時間間隔は,ボタン押し課題を行っている間,電子メトロノームの合図音によって提示する.
そして,被験者は聴覚フィードバックに遅延が発生している状態で本研究のボタン押し課題を行う.
遅延聴覚フィードバックが身体運動に何らかの影響を与えていれば,被験者がボタンを押下する時間間隔にばらつきが発生すると考えられる.
そこで,被験者がボタンを押下する時間間隔の全体の分散およびボタンの押下回数が4の倍数に到達する直前の押下時間間隔と4の倍数に到達した直後の押下時間間隔の平均二乗誤差を評価指標として用いる.
また,遅延によるボタン押し課題への影響が大きく現れる最適な実験条件を探るために,ボタン押し課題における最適なボタンの押下時間間隔,ボタン押し課題の長さおよび最適な遅延のタイミングについて検討する.
そして,決定した条件の下で,聴力が正常な若年者と高齢者を対象に遅延聴覚フィードバックが身体運動に与える影響について調査を行う.本研究は,聴覚フィードバックの遅延が
人の身体運動にどのような影響を与えるか,特に年齢差がその影響にどのようにして関わってくるかを明らかにすることに寄与することが期待される.
加えて,主観的な評価方法に基づく遅延聴覚フィードバックが発話に与える影響の調査も継続するため,そのためのアプリケーションの改善も行う.
本研究の成果は,高齢者向けの補聴器の設計において,重要な示唆を提供するものであると考えられる.
\newpage