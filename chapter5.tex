\chapter{遅延聴覚フィードバックが身体運動に与える影響の評価方法}
本章では,遅延聴覚フィードバックが身体運動に与える影響の評価方法を述べる.
4章で説明した方法で被験者がボタンを押下する時間を記録すると,被験者に提示するボタンの押下の時間間隔が毎分60回であれば,理想的には1000[ms]の間隔でボタンが押下される.
しかし,人がボタンを押下すると理想的なボタン押下の時間間隔にばらつきが生じると考えられる.
また,遅延聴覚フィードバックが身体運動に影響を与えて入れば,このばらつきは聴覚フィードバックの遅延時間の大きさによって変化するものであると考えらえる.
そこで,このボタン押下の時間間隔のばらつきを各遅延時間で評価することで,遅延聴覚フィードバックが身体運動に与える影響を評価する.
ばらつきの評価には,分散と二乗誤差を用いる.これらの評価指標は,ばらつきが大きくなるほど値が大きくなるため,評価指標の計算結果の大小でばらつきの大きさを観察することが可能になると考えられる.
そして,これらの評価指標を算出するとき,評価指標の値の時間変化や評価に使用するデータの個数によって値が変化を比較することを想定して,これらをパラメータとして用いる.
\section{分散}
まず、ボタンを押下する時間間隔の不偏分散$s^2_{1}$は,被験者が行うボタン押下の時間間隔を用いて算出する.$s^2_{1}$は,以下の式により示される.
\begin{equation}
  s^2_a = \frac{1}{l-1} \sum_{i=k}^{k+l-1} (x_i - \bar{x}_{kl})^2
\end{equation}
ここで$l$[回]は分散を算出するために使用するデータの個数,$k$は分析するデータの最初のインデックス,$x_{i}$[ms]は,取得した$i$番目のボタンの押下時間間隔のデータ,$\bar{x}_{kl}$[ms]は,$k$番目のデータから$n$個のデータを用いて算出するボタンの押下時間間隔のデータの平均値を指す.
$s^2_{1}$において,任意の$i$で理想的な時刻にボタンが押下されなかった場合,$x_{i}$と$x_{i+1}$の両方に理想的なボタンの押下時間間隔との差異が生じる.データの平均値との差異を分析する際,大きな誤差が発生した場合,その影響で分散が過大になり,適切な評価が困難になる可能性がある.したがって,各被験者のボタンの押下時間間隔のデータの中央値を真値とする分散を検討することが有効である.中央値を真値として用いることにより,データに極端な誤差が生じた場合でも、ボタンの押下時間間隔のばらつきを適切に評価することが可能になると考えられる.各被験者のボタンの押下時間間隔のデータの中央値を真値とする分散$s^2_{2}$は,以下の式により示される.
\begin{equation}
  s^2_b = \frac{1}{l} \sum_{i=k}^{k+l-1} (x_i - M_{kl})^2
\end{equation}
を用いてここで,$M_{kl}$は,$k$番目のデータから$n$個のデータを用いて算出するボタンの押下時間間隔のデータの中央値を指す.
次に,真値を理想的なボタンの押下時間間隔とする場合を考える.例えば,ボタン押下の回数が毎分60回であれば,理想的には1000[ms]の間隔でボタンが押下される.しかし,実際の実験では,この理想的な間隔でボタンが押下されるとは考えにくく,その理想的な間隔との誤差の分散は,ばらつきが増加するとともに大きくなると推測される.理想的なボタンの押下時間間隔との誤差の分散$s^2_{3}$は,以下の式により示される.
\begin{equation}
  s^2_c = \frac{1}{l} \sum_{i=k}^{k+l-1} (x_i - T)^2
\end{equation}
% \section{標準偏差}
% まず,ボタンを押下する時間間隔の標準偏差$s_{1}$[ms]は,記録するボタンを押下する時間間隔を用いて算出する.$s_{1}$は次式で表される.
% \begin{equation}
%   s_1 = \sqrt{\frac{1}{n-1} \sum_{i=k}^{k+n-1} (x_i - \bar{x}_{kl})^2}
% \end{equation}
% \begin{equation}
%   s_2 = \sqrt{\frac{1}{n} \sum_{i=k}^{k+n-1} (x_i - M)^2}
% \end{equation}
% \begin{equation}
%   s_3 = \sqrt{\frac{1}{n} \sum_{i=k}^{k+n-1} (x_i - T)^2}
% \end{equation}
\section{二乗誤差}
本節では,聴覚フィードバックの遅延が変則的に発生する場合のばらつき評価について検討する.遅延が$t$の倍数に到達した際のみ発生する状況を想定し,$t$の倍数に到達する直前のボタン押下間隔と直後のボタン押下間隔のデータの差の二乗平均(Mean Squared Error, MSE)が,遅延聴覚フィードバックがボタン押下間隔に与える影響を反映すると仮定する.遅延時間が増加するにつれて,MSEも増加すると予測される.
MSEを算出する際に用いる誤差の総数を表す関数$f(n)$と,使用するデータの最後のインデックスを示す関数$s(n)$は,ボタン押下回数$n$を用いて以下の式で表される.

\begin{equation}
f(n) = \left\lfloor \frac{n-t-1}{t} \right\rfloor + 1
\end{equation}
\begin{equation}
s(n) = \left\lfloor \frac{n-t-2}{t-1} \right\rfloor
\end{equation}

ここで,$\left\lfloor x \right\rfloor$は$x$を超えない最大の整数を示す.

これらの関数を用いて,MSEは次の式で定義される.

\begin{equation}
MSE = \frac{1}{f(n)} \sum_{i=0}^{s(n)} (d_{t-1+ti} - d_{t+ti})^2
\end{equation}

ここで,$d_{t-1+ti}$は$t$の倍数に達する直前のボタン押下間隔,$d_{t+ti}$は$t$の倍数に達した直後のボタン押下間隔のデータを示す.これにより,聴覚フィードバックの遅延による影響を定量的に評価することが可能となると考えられる.
さらに,$t$の倍数に到達する直前のボタン押下時間間隔と到達した直後の間隔との誤差の中央値(Median Squared Error, MedSE)での評価を検討する.この計算法により,誤差の極端な値が存在しても適切なばらつきの評価が行える可能性がある.誤差の中央値$MedSE$は以下の式で定義される.

\begin{equation}
MedSE = Med((d_{3}-d_{4})^2, (d_{7}-d_{8})^2, \ldots, (d_{t-1+ts(n)}-d_{t+ts(n)})^2)
\end{equation}

この式では,$Med()$は中央値を計算する関数であり,括弧内の各項は$t$の倍数に到達する直前後のボタン押下間隔の差の二乗を示す.このように,$MedSE$を用いることで,データの極端なばらつきによる影響を抑えつつ,遅延聴覚フィードバックの効果をより適切に評価することが期待される.

% \section{絶対値誤差}
% \begin{equation}
%   MAE = \frac{1}{f(n)} \sum_{i=0}^{s(n)} |d_{3+4i} - d_{4+4i}|
% \end{equation}
