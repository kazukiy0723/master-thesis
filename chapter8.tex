\chapter{結論}
\section{まとめ}
%%%%% 主観調査のためのアプリ開発 %%%%%
本研究では,先行研究から継続して実施予定の遅延聴覚フィードバックが発話に及ぼす影響の調査に向け,アプリケーションを開発した.
このアプリケーションは,被験者がアプリケーション上に直接評価結果を入力でき,その結果を外部ファイルに出力する機能を備えている.
この開発したアプリケーションを用いることにより,実験の効率化および評価結果の分析が容易になることが期待される.

%%%%%% 客観調査%%%%%%
加えて,聴覚フィードバックによる違和感を客観的に評価するために,先行研究\cite{shigematu}で開発されたシステムを改良し,また,ボタン押し課題を用いた客観的な評価方法を開発した.
そして,これらを用いて遅延聴覚フィードバックの身体運動への影響を調査した.
開発したボタン押し課題では,被験者は一定の時間間隔でボタンを押下し,ヘッドホンからの聴覚フィードバックの遅延がボタンの押下回数が4の倍数に達したときのみ発生するように設定した.また,
この方法において,遅延聴覚フィードバックの影響によるボタンの押下時間間隔のばらつきを観察しやすくなるよう,客観的な評価指標および調査条件を定めた.
その結果,ボタン押し課題では,ボタンを押下する間隔を1分間に80回とし,
客観的な評価指標としてボタン押下間隔の平均および中央値を真値とする分散,
およびボタンの押下回数が4の倍数に到達する直前の押下間隔と4の倍数に到達した直後の押下間隔の平均二乗誤差および中央値二乗誤差を用いることが適切であると判断した.

そして,聴力の正常な若年者と高齢者を対象に,本研究で開発したボタン押し課題を用いて遅延聴覚フィードバックの身体運動への影響を調査した.
その結果,10[ms]から40[ms]の短い遅延時間では,若年者の評価値は遅延時間の増加に伴って,緩やかに増加する傾向が見られたが,
高齢者の評価値には一貫した関係が見出されなかった.
これは,若年者が遅延に敏感である一方で,高齢者が遅延時間に対して一定の許容度を持っていることを示している.
遅延時間が10[ms]から110[ms]におよぶ実験では,高齢者が遅延時間の増加に対して比較的鈍感であるものの,
一定の遅延時間を超えるとタスクの一貫性を保つことが困難になることが明らかになった.
これらの結果は,聴覚フィードバックの遅延に対する年齢別の感受性の違いを浮き彫りにし,高齢者向けの補聴器設計における重要な知見を提供するものであると考えられる.
さらに,高齢者が若年者に比べてすべての遅延時間で高い評価値を示したという結果は,高齢者と若年者間の潜在的な運動能力の差異がこれらの反応に影響している可能性を示唆している.
\section{今後の展望}
%%%%% 以下は,今後の展望の例 %%%%%
% ・3の倍数でもよくないか?\\
% ・聴力に異常がある人とない人では差があるのか?\\
% ・より細かい遅延時間で調査する必要があるのか?\\
% ・潜在的な身体能力の影響を軽減させるような課題を考える必要がある.
% 違和感を感じることと身体運動に影響があることは別のことである可能性がある.
% 客観調査と主観調査を同時に行うことで,遅延聴覚フィードバックが身体運動に与える影響をより詳細に調査することができると考えられる.
今後は,高齢者と若年者の運動能力の差異を調整して遅延聴覚フィードバックの影響をより公平に評価するために,被験者の運動能力を考慮した課題を検討する必要がある.
その一例として,運動能力の基本的な側面を評価するためのタスクを導入し,参加者の反応速度や運動の一貫性を測定することで,運動能力の基準値を
設定し,遅延時間に対する反応の差異を適切に解釈することができるようになると考えれられる.
さらに,課題の難易度を調整することで,運動能力以外の要因が遅延聴覚フィードバックの影響をどのように受けるかを評価し,運動能力以外の要素が
影響を与えるかどうかを調査することが可能になると考えられる.
また,遅延聴覚フィードバックが発話に及ぼす影響の客観的な評価方法の検討および本研究で得られたデータとの比較も必要である.
これらのアプローチを通じて,遅延聴覚フィードバックの影響をより深く理解し,それが補聴器の設計に繋がることが期待される.