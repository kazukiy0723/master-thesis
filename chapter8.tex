\chapter{結論}
\section{まとめ}
本研究では,聴覚フィードバックによる違和感を客観的に評価するために,先行研究のシステムを改良して,遅延聴覚フィードバックの
身体運動への影響をボタン押し課題で調査した.このボタン押し課題は,
一定の時間間隔ごとに一定回数ボタンを押すという課題である.被験者がボタンを押下する間隔は,
電子メトロノームによる合図音によって規定される.
ボタンを押下したときに,ヘッドホンから聴覚フィードバックとして音が出力される.
この音をボタンの押下回数が4の倍数に達したときのみ遅延させることで,遅延聴覚フィードバックの影響を調査した.
そして,聴覚フィードバックの遅延による影響によるボタンの押下間隔のばらつきが観察しやすくなるように,
客観的な評価指標および調査条件を決定するための実験を行った.その結果から,ボタン押し課題では,
ボタンを押下する間隔を1分間に80回,客観的な評価に用いる指標をボタン押下間隔の平均および中央値を真値とする分散,ならびに
ボタンの押下回数が4の倍数に到達する直前の押下間隔と4の倍数に到達した直後の押下間隔の平均二乗誤差および中央値二乗誤差
とすることが適切であることが示唆された.
そして,聴力の正常な若年者と高齢者を対象に,ボタン押し課題を用いて遅延聴覚フィードバックの身体運動への影響を調査した.
その結果,高齢者は若年者と比較して,聴覚フィードバックの遅延に対する許容度が高い可能性が示唆された.


\section{今後の展望}
